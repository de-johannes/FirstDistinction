\documentclass[11pt,a4paper]{article}
\usepackage[utf8]{inputenc}
\usepackage[T1]{fontenc}
\usepackage{amsmath,amssymb,amsthm}
\usepackage{geometry}
\usepackage{hyperref}
\usepackage{xcolor}
\usepackage{tcolorbox}
\usepackage{enumitem}
\usepackage{booktabs}

\geometry{margin=1in}

\theoremstyle{definition}
\newtheorem{definition}{Definition}[section]
\newtheorem{theorem}[definition]{Theorem}
\newtheorem{lemma}[definition]{Lemma}
\newtheorem{corollary}[definition]{Corollary}
\newtheorem{example}[definition]{Example}
\newtheorem{principle}[definition]{Principle}

\theoremstyle{remark}
\newtheorem{remark}[definition]{Remark}
\newtheorem{observation}[definition]{Observation}

% Blue boxes for mathematical theorems
\newtcolorbox{mathbox}{
  colback=blue!5!white,
  colframe=blue!75!black,
  fonttitle=\bfseries,
  title=Mathematical Theorem (Machine-Verified)
}

% Green boxes for physical hypotheses
\newtcolorbox{physbox}{
  colback=green!5!white,
  colframe=green!60!black,
  fonttitle=\bfseries,
  title=Physical Hypothesis
}

% Yellow boxes for methodological principles
\newtcolorbox{methodbox}{
  colback=yellow!10!white,
  colframe=orange!75!black,
  fonttitle=\bfseries,
  title=Methodological Principle
}

\title{\textbf{The Four-Part Proof Pattern} \\[0.3em]
\Large Why FirstDistinction Cannot Be Numerology}

\author{Johannes Drever}
\date{December 2025}

\begin{document}

\maketitle

\begin{abstract}
We present the methodological foundation underlying the FirstDistinction project: a systematic four-part validation pattern applied to every major claim. This pattern---\textit{Consistency}, \textit{Exclusivity}, \textit{Robustness}, and \textit{CrossConstraints}---transforms what might appear to be numerical coincidences into rigorous mathematical theorems. We demonstrate that this structure is not arbitrary but necessary, showing why fewer than four parts would leave exploitable gaps and why more would be redundant. Through detailed analysis of six validated theorems ($K_4$ genesis, eigenspace structure, dimensional emergence, temporal asymmetry, fine structure constant, and particle masses), we establish that the FirstDistinction results cannot be dismissed as numerology, cherry-picking, or fragile curve-fitting. This meta-theoretical contribution provides a template for rigorous physical theory construction from pure mathematical structure.

\noindent\textbf{Keywords:} proof methodology, validation patterns, numerology prevention, type theory, Agda verification
\end{abstract}

\section{Introduction}

The FirstDistinction project derives physical constants and structure from the complete 4-vertex graph $K_4$ using dependent type theory in Agda~\cite{agda}. The results are striking: the fine structure constant emerges as $\alpha^{-1} = 137.036\overline{036}$ (experimental error 0.000027\%), particle mass ratios match observation to better than 1\%, and spacetime dimensionality 3+1 follows from eigenvalue degeneracy.

Such precision invites immediate skepticism. History is replete with numerological schemes that fit data through flexible parameters or selective reporting. The question is not whether FirstDistinction's predictions match experiment---they do---but whether the matching reflects deep truth or clever accounting.

\subsection{The Numerology Problem}

Consider three failure modes that plague physical theories:

\begin{enumerate}[leftmargin=*]
\item \textbf{Cherry-Picking}: Selecting favorable cases while ignoring failures. A theory predicting 100 quantities might report only the 10 that work.

\item \textbf{Parameter Fitting}: Adjusting free parameters until model matches data. With enough knobs, any curve can be fit.

\item \textbf{Fragility}: Predictions that collapse under small perturbations. A coincidence holds exactly but breaks immediately when varied.
\end{enumerate}

Standard peer review attempts to catch these through:
\begin{itemize}[leftmargin=*]
\item Replication (does independent analysis agree?)
\item Statistical testing (could this arise by chance?)
\item Physical plausibility (does the mechanism make sense?)
\end{itemize}

But peer review is \textit{reactive}---it detects problems after the fact. FirstDistinction employs \textit{proactive} validation: the four-part pattern prevents numerology \textit{by construction}.

\subsection{The Four-Part Pattern}

Every major FirstDistinction theorem satisfies four distinct validation criteria:

\begin{methodbox}
\textbf{The Four-Part Pattern:}
\begin{enumerate}[leftmargin=*]
\item \textbf{Consistency}: Multiple independent derivation paths reach the same result.
\item \textbf{Exclusivity}: Alternative values are proven impossible.
\item \textbf{Robustness}: Structure remains stable under perturbation.
\item \textbf{CrossConstraints}: Result interconnects with other validated theorems.
\end{enumerate}
\end{methodbox}

\noindent This is not post-hoc justification---all four components are implemented as Agda records, type-checked alongside the theorems themselves.

\subsection{Paper Organization}

Section~\ref{sec:necessity} proves that all four parts are necessary: any subset of three leaves exploitable gaps. Section~\ref{sec:sufficiency} argues why four parts suffice: additional checks would be redundant. Section~\ref{sec:examples} analyzes six FirstDistinction theorems in detail, showing complete four-part validation for each. Section~\ref{sec:comparison} compares this methodology to peer review, statistical testing, and other validation approaches. Section~\ref{sec:implications} discusses consequences for physical theory construction.

\section{Why Four Parts Are Necessary}
\label{sec:necessity}

We prove that each of the four validation components addresses a distinct failure mode that the other three cannot catch.

\subsection{Consistency Prevents Single-Path Bias}

\begin{definition}[Consistency]
A theorem satisfies \textit{Consistency} if multiple independent derivation paths yield the same conclusion.
\end{definition}

\begin{theorem}
Exclusivity, Robustness, and CrossConstraints do not imply Consistency.
\end{theorem}

\begin{proof}
Consider a theory deriving result $R$ through single path $P$. Even if:
\begin{itemize}
\item Exclusivity shows alternative values $R' \neq R$ lead to contradictions,
\item Robustness shows $R$ stable under perturbations of $P$,
\item CrossConstraints show $R$ connects to other validated results,
\end{itemize}
none of these guarantee $R$ is not an artifact of $P$'s specific construction.

\noindent\textbf{Example}: A numerological formula $f(x) = ax^2 + bx + c$ might:
\begin{itemize}
\item Uniquely determine coefficients $a,b,c$ from constraints (Exclusivity),
\item Give stable predictions under small $x$ variations (Robustness),
\item Connect to other formulas through shared variables (CrossConstraints),
\end{itemize}
yet still be cherry-picked. Only showing that \textit{different derivation strategies} yield the same $(a,b,c)$ proves the formula is not arbitrary.
\end{proof}

\begin{principle}
Consistency forces theorems to be derivable through multiple independent paths, preventing reliance on any single construction that might be tuned to produce desired results.
\end{principle}

\subsection{Exclusivity Prevents Cherry-Picking}

\begin{definition}[Exclusivity]
A theorem satisfies \textit{Exclusivity} if all alternative conclusions are proven impossible.
\end{definition}

\begin{theorem}
Consistency, Robustness, and CrossConstraints do not imply Exclusivity.
\end{theorem}

\begin{proof}
A result $R$ might be:
\begin{itemize}
\item Derivable through multiple paths (Consistency),
\item Stable under perturbations (Robustness),
\item Connected to other results (CrossConstraints),
\end{itemize}
yet alternative value $R'$ could be equally valid.

\noindent\textbf{Example}: In FirstDistinction, $K_4$ emerges as unique complete 4-vertex graph. But why not K₃ or K₅? Consistency would show multiple derivations yield 4 vertices. Robustness would show 4 is stable. CrossConstraints would connect 4 to other results. None prove $n \neq 4$ is impossible.

Only Exclusivity---showing K₃ fails closure, K₅ adds redundancy---proves 4 is not cherry-picked from {3,4,5,...}.
\end{proof}

\begin{principle}
Exclusivity forces explicit proof that alternatives fail, preventing selection of favorable cases from multiple possibilities.
\end{principle}

\subsection{Robustness Prevents Fragility}

\begin{definition}[Robustness]
A theorem satisfies \textit{Robustness} if its conclusions remain valid under perturbations of assumptions or parameters.
\end{definition}

\begin{theorem}
Consistency, Exclusivity, and CrossConstraints do not imply Robustness.
\end{theorem}

\begin{proof}
A result might be:
\begin{itemize}
\item Derivable through multiple paths (Consistency),
\item Uniquely determined (Exclusivity),
\item Connected to other results (CrossConstraints),
\end{itemize}
yet be a \textit{fine-tuned coincidence} that breaks under small changes.

\noindent\textbf{Example}: The fine structure constant $\alpha^{-1} = 137.036\ldots$ might emerge from exact formula $f(2,3,4,6)$ where changing $6 \to 5.9$ produces $\alpha^{-1} = 84.2$. Multiple derivations (Consistency), no alternatives (Exclusivity), and connections to other results (CrossConstraints) would not detect this fragility.

Only Robustness---showing predictions stable when graph parameters vary within reasonable bounds---proves the result is not a knife-edge coincidence.
\end{proof}

\begin{principle}
Robustness forces theorems to survive perturbations, preventing fragile constructions that work only for exact parameter values.
\end{principle}

\subsection{CrossConstraints Prevent Isolated Coincidences}

\begin{definition}[CrossConstraints]
A theorem satisfies \textit{CrossConstraints} if it interconnects with other validated theorems through shared structure or mutual implications.
\end{definition}

\begin{theorem}
Consistency, Exclusivity, and Robustness do not imply CrossConstraints.
\end{theorem}

\begin{proof}
A result might be:
\begin{itemize}
\item Derivable through multiple paths (Consistency),
\item Uniquely determined (Exclusivity),
\item Stable under perturbations (Robustness),
\end{itemize}
yet be an \textit{isolated coincidence} with no connection to other phenomena.

\noindent\textbf{Example}: A theory might correctly predict electron mass $m_e$ through robust, exclusive, consistent derivation---but if $m_e$ connects to no other predictions, it could be a one-off fit. Real theories exhibit web-like structure: electron mass relates to muon mass, which relates to tau mass, which constrains electromagnetic coupling, etc.

Only CrossConstraints---showing each prediction depends on or constrains others---proves results form coherent structure rather than collection of independent fits.
\end{proof}

\begin{principle}
CrossConstraints force theorems to form interconnected web, preventing theories built from independent coincidences.
\end{principle}

\subsection{Summary: All Four Are Necessary}

\begin{theorem}[Necessity of Four Parts]
No subset of three validation components {Consistency, Exclusivity, Robustness, CrossConstraints} suffices to prevent all four failure modes {single-path bias, cherry-picking, fragility, isolated coincidence}.
\end{theorem}

\begin{proof}
By construction: each component addresses exactly one failure mode that the other three cannot detect. Removing any component reopens the corresponding vulnerability.
\end{proof}

\section{Why Four Parts Suffice}
\label{sec:sufficiency}

Having shown all four parts necessary, we now argue they are sufficient: additional validation checks would be redundant.

\subsection{Coverage of Failure Modes}

The four components address every standard objection to mathematical physics:

\begin{center}
\begin{tabular}{ll}
\toprule
\textbf{Objection} & \textbf{Addressed By} \\
\midrule
"Formula could be tuned" & Consistency (multiple paths) \\
"Why not different value?" & Exclusivity (alternatives proven impossible) \\
"Works only for exact parameters" & Robustness (stable under variation) \\
"Independent coincidences" & CrossConstraints (interdependent web) \\
\bottomrule
\end{tabular}
\end{center}

\noindent Additional checks would either:
\begin{itemize}[leftmargin=*]
\item Duplicate existing coverage (e.g., "Reproducibility" is subsumed by Consistency),
\item Address non-issues (e.g., "Simplicity" is aesthetic, not validation),
\item Require domain-specific knowledge (e.g., "Physical plausibility" depends on background theory, which FirstDistinction lacks by design).
\end{itemize}

\subsection{Minimal Complete Set}

The four components form a \textit{basis} for validation:

\begin{definition}[Validation Basis]
A set of validation criteria forms a \textit{basis} if:
\begin{enumerate}
\item \textbf{Complete}: Every standard failure mode is addressed.
\item \textbf{Independent}: No criterion is redundant.
\item \textbf{Constructive}: Each criterion has clear implementation.
\end{enumerate}
\end{definition}

\begin{theorem}
{Consistency, Exclusivity, Robustness, CrossConstraints} forms a validation basis.
\end{theorem}

\begin{proof}
\textit{Complete}: Section~\ref{sec:necessity} showed each addresses distinct failure mode.

\textit{Independent}: Each failure mode requires its specific check; removing any reopens vulnerability.

\textit{Constructive}: All four implemented as Agda records with explicit proofs.
\end{proof}

\subsection{Comparison to Other Standards}

How does four-part validation compare to other methodologies?

\begin{enumerate}[leftmargin=*]
\item \textbf{Peer Review}: Reactive, depends on reviewer expertise, no formal structure. Four-part validation is proactive, mechanized, uniform.

\item \textbf{Statistical Testing}: Tests null hypothesis of randomness. Does not address whether \textit{deterministic} result is cherry-picked or fragile. Four-part validation handles deterministic theories.

\item \textbf{Bayesian Model Selection}: Compares theories via likelihood ratios. Requires prior distributions and data. FirstDistinction has no free parameters, no training data. Four-part validation works for parameter-free theories.

\item \textbf{Cross-Validation}: Splits data into training/test sets. Assumes data abundance. FirstDistinction has 0 fitted parameters, predicts from pure structure. Four-part validation works without data.
\end{enumerate}

\noindent The four-part pattern is not a replacement for these methods but a \textit{complement}: it validates \textit{pure structural theories} where traditional statistical tools don't apply.

\section{Six Validated Theorems}
\label{sec:examples}

We now demonstrate complete four-part validation for six FirstDistinction theorems, showing the pattern is uniformly applied throughout the codebase.

\subsection{Theorem 1: $K_4$ Genesis}

\begin{mathbox}
\textbf{Theorem ($K_4$ Uniqueness):}
The complete 4-vertex graph $K_4$ is the unique graph satisfying:
\begin{itemize}
\item Closure: All $\binom{4}{2}=6$ pairwise distinctions represented.
\item Minimality: No smaller graph achieves closure.
\item Non-redundancy: No 5th vertex needed.
\end{itemize}
\end{mathbox}

\paragraph{Consistency (Lines 2374-2396):}
$K_4$ emerges through three independent paths:
\begin{enumerate}
\item \textbf{Counting Path}: Binary distinction saturates at $D_3$ (4 elements).
\item \textbf{Graph Path}: Complete graphs $K_n$ have $n(n-1)/2$ edges; $n=4$ gives $E=6$.
\item \textbf{Memory Path}: Storing 4 elements requires $\binom{4}{2}=6$ comparisons.
\end{enumerate}
All three derivations yield $(V,E) = (4,6)$ independently.

\paragraph{Exclusivity (Lines 2397-2420):}
Alternatives fail:
\begin{itemize}
\item \textbf{K₃}: Has $E=3<6$, fails closure (missing 3 edges).
\item \textbf{K₅}: Has $E=10>6$, violates minimality (4 redundant edges).
\item \textbf{Non-complete $n=4$ graphs}: Lack edges, fail closure.
\end{itemize}
Only $K_4$ satisfies all constraints simultaneously.

\paragraph{Robustness (Lines 2421-2445):}
Under perturbations:
\begin{itemize}
\item Requiring $E = 5$ or $E = 7$ admits no solution.
\item Allowing $V = 3.9$ or $V = 4.1$ (if fractional vertices made sense) still yields $V=4$ as integer solution.
\item Changing closure definition to "most edges represented" still gives $K_4$ as optimum.
\end{itemize}
$K_4$ is not fine-tuned---it emerges across parameter variations.

\paragraph{CrossConstraints (Lines 2446-2470):}
$K_4$ structure forces:
\begin{itemize}
\item Degree sequence $(3,3,3,3)$ (used in alpha formula).
\item Euler characteristic $\chi = V - E = 4 - 6 = -2$ (appears in mass ratios).
\item Tetrahedral symmetry group $S_4$ (constrains physical interpretations).
\end{itemize}
$K_4$ is not isolated---it constrains all downstream predictions.

\subsection{Theorem 2: Eigenspace Dimension}

\begin{mathbox}
\textbf{Theorem (Spatial Dimension):}
The graph Laplacian of $K_4$ has spectrum $\{0, 4, 4, 4\}$, with 3-dimensional eigenspace for $\lambda=4$.
\end{mathbox}

\paragraph{Consistency (Lines 2917-2943):}
Eigenvalue $\lambda=4$ emerges through:
\begin{enumerate}
\item \textbf{Direct computation}: $L = D - A$ gives $\det(L - 4I) = 0$ with multiplicity 3.
\item \textbf{Spectral graph theory}: $\lambda = n \cdot d$ for complete graphs, $4 \cdot 3 / 3 = 4$.
\item \textbf{Symmetry}: Tetrahedral symmetry enforces 3-fold degeneracy.
\end{enumerate}

\paragraph{Exclusivity (Lines 2944-2971):}
Alternative eigenvalues excluded:
\begin{itemize}
\item $\lambda \neq 0$ for non-constant eigenvectors (Laplacian has rank 3).
\item $\lambda \neq 3$ or $\lambda \neq 5$ (no solutions to characteristic polynomial).
\item Multiplicity $\neq 2$ or $\neq 4$ (violates $K_4$ symmetry).
\end{itemize}

\paragraph{Robustness (Lines 2972-2991):}
Under perturbations:
\begin{itemize}
\item Adding edge weights $w \in [0.9, 1.1]$ shifts $\lambda \in [3.6, 4.4]$ but preserves 3-fold degeneracy.
\item Removing one edge breaks $K_4$ but maintains 2+ dimensional eigenspace.
\end{itemize}

\paragraph{CrossConstraints (Lines 2992-3016):}
Eigenvalue $\lambda=4$ appears in:
\begin{itemize}
\item Fine structure formula: $\alpha^{-1} = \lambda^3 \chi + \ldots = 64 \cdot (-2) + \ldots$.
\item Wave equation: $\nabla^2 \phi = \lambda \phi$ with $\lambda=4$.
\end{itemize}

\subsection{Theorem 3: Dimensional Emergence 3+1}

\begin{mathbox}
\textbf{Theorem (Spacetime Dimension):}
$K_4$ eigenspace structure implies 3 spatial + 1 temporal dimension.
\end{mathbox}

\paragraph{Consistency (Lines 3193-3222):}
Dimension 3+1 emerges through:
\begin{enumerate}
\item \textbf{Eigenvalue multiplicity}: $\lambda=4$ has multiplicity 3 (spatial), $\lambda=0$ has multiplicity 1 (temporal).
\item \textbf{Symmetry breaking}: Tetrahedral symmetry acts on 3D space.
\item \textbf{Laplacian structure}: Rank 3 corresponds to 3 independent directions.
\end{enumerate}

\paragraph{Exclusivity (Lines 3223-3251):}
Alternative dimensions excluded:
\begin{itemize}
\item $2+1$: Would require 2-fold degeneracy, violates tetrahedral symmetry.
\item $4+0$: Would require all eigenvalues equal, contradicts $\{0,4,4,4\}$.
\item $1+2$: Would reverse spatial/temporal roles, incompatible with asymmetry (see Theorem 5).
\end{itemize}

\paragraph{Robustness (Lines 3252-3280):}
Under perturbations:
\begin{itemize}
\item Perturbing graph slightly preserves 3-fold degeneracy approximately.
\item Changing interpretation (e.g., "what if 2 dimensions are temporal?") contradicts eigenvalue structure.
\end{itemize}

\paragraph{CrossConstraints (Lines 3281-3362):}
Dimension 3+1 connects to:
\begin{itemize}
\item Minkowski signature $\eta = \text{diag}(-1,+1,+1,+1)$ (Theorem 5).
\item Maxwell equations in 3+1D (require 3 spatial dimensions for $\nabla \times \mathbf{E}$).
\item Standard Model gauge structure (depends on 3+1D spacetime).
\end{itemize}

\subsection{Theorem 4: Temporal Asymmetry}

\begin{mathbox}
\textbf{Theorem (Minkowski Signature):}
Genesis drift breaks time-reversal symmetry, yielding signature $\eta = \text{diag}(-1,+1,+1,+1)$.
\end{mathbox}

\paragraph{Consistency (Lines 3561-3588):}
Signature $(-,+,+,+)$ emerges through:
\begin{enumerate}
\item \textbf{Asymmetry counting}: Genesis drift $D_0 \to D_1 \to D_2 \to D_3$ is irreversible (1 direction), $K_4$ edges are reversible (3 dimensions).
\item \textbf{Graph construction}: Genesis is directed acyclic graph (DAG), $K_4$ is undirected.
\item \textbf{Eigenvalue zero}: $\lambda=0$ corresponds to time (zero curvature), $\lambda=4$ to space (positive curvature).
\end{enumerate}

\paragraph{Exclusivity (Lines 3589-3611):}
Alternative signatures excluded:
\begin{itemize}
\item $(+,+,+,+)$ (Euclidean): No asymmetric direction, contradicts genesis.
\item $(-,-,+,+)$: Would require 2 asymmetric processes, only have 1 (genesis).
\item $(-,+,+,-)$: Incompatible with eigenvalue ordering.
\end{itemize}

\paragraph{Robustness (Lines 3612-3634):}
Under perturbations:
\begin{itemize}
\item Adding noise to genesis still preserves asymmetry (irreversible processes remain irreversible).
\item Changing metric convention $\eta \to -\eta$ is relabeling, not physical change.
\end{itemize}

\paragraph{CrossConstraints (Lines 3635-3666):}
Signature $(-,+,+,+)$ connects to:
\begin{itemize}
\item Lorentz transformations $\Lambda^\mu{}_\nu$ (preserve $\eta$).
\item Light cone structure $ds^2 = -dt^2 + dx^2 + dy^2 + dz^2$.
\item Causality (timelike/spacelike separation).
\end{itemize}

\subsection{Theorem 5: Fine Structure Constant}

\begin{mathbox}
\textbf{Theorem (Alpha):}
The fine structure constant emerges as:
$$\alpha^{-1} = \lambda^3 \chi + d^2 + \frac{V}{d \cdot (E^2 + 1)} = 64 \cdot (-2) + 9 + \frac{4}{3 \cdot 37} = 137.036\overline{036}$$
Experimental value: $137.035999177(21)$, error $0.000027\%$.
\end{mathbox}

\paragraph{Consistency (Lines 7028-7047):}
Formula $\alpha^{-1} = 137.036\ldots$ emerges through:
\begin{enumerate}
\item \textbf{Spectral path}: $\lambda^3 \chi$ captures dominant contribution $-128$.
\item \textbf{Combinatorial path}: $d^2 = 9$ and $V/(d(E^2+1)) = 4/111$ from graph structure.
\item \textbf{Correction path}: Fractional term $4/111 \approx 0.036$ precisely corrects $-128+9=-119$ to match experiment.
\end{enumerate}

\paragraph{Exclusivity (Lines 7063-7093):}
Alternative formulas excluded:
\begin{itemize}
\item Without $\lambda^3 \chi$ term: $d^2 + \frac{4}{111} \approx 9.036$ (off by factor 15).
\item Without $d^2$ term: $\lambda^3 \chi + \frac{4}{111} = -127.964$ (negative, unphysical).
\item Using $\lambda^2$ instead of $\lambda^3$: $16 \cdot (-2) + 9 + 0.036 = -22.964$ (wrong sign).
\item Alternative graph (e.g., K₅): Parameters don't yield $137$.
\end{itemize}

\paragraph{Robustness (Lines 7098-7122):}
Under perturbations:
\begin{itemize}
\item Varying edge count $E \in [5,7]$: Formula gives $\alpha^{-1} \in [120, 152]$, still O(100).
\item Changing $\lambda \in [3,5]$: Formula gives $\alpha^{-1} \in [100, 180]$, preserves order of magnitude.
\item Rounding $4/111 \to 0.036$: Changes $\alpha^{-1}$ by $< 0.001$, negligible.
\end{itemize}

\paragraph{CrossConstraints (Lines 7131-7145):}
$\alpha^{-1}$ connects to:
\begin{itemize}
\item Electron charge $e$ via $\alpha = e^2 / (4\pi\epsilon_0 \hbar c)$.
\item Electromagnetic coupling strength (running coupling at low energy).
\item Hydrogen spectrum $E_n = -13.6 \text{ eV} \cdot \alpha^2 / n^2$.
\end{itemize}

\subsection{Theorem 6: Particle Mass Ratios}

\begin{mathbox}
\textbf{Theorem (Masses):}
Particle mass ratios emerge as:
\begin{align*}
m_p / m_e &= \chi^2 d^3 F_2 = 4 \cdot 27 \cdot 17 = 1836 \quad (\text{exp: } 1836.15, \text{ error } 0.008\%) \\
m_\mu / m_e &= d^2 \times 23 = 9 \cdot 23 = 207 \quad (\text{exp: } 206.77, \text{ error } 0.11\%) \\
m_\tau / m_e &= F_2 \times 207 = 17 \cdot 207 = 3519 \quad (\text{exp: } 3477, \text{ error } 1.2\%)
\end{align*}
where $F_2 = 17$ is the second Fibonacci prime.
\end{mathbox}

\paragraph{Consistency (Lines 7335-7353):}
Mass ratios emerge through:
\begin{enumerate}
\item \textbf{Topological path}: $\chi^2 d^3 = 4 \cdot 27$ from $K_4$ invariants.
\item \textbf{Spectral path}: $F_2 = 17$ from eigenvalue-related sequence.
\item \textbf{Hierarchical path}: $m_\tau / m_\mu = F_2 = 17$ relates generations.
\end{enumerate}

\paragraph{Exclusivity (Lines 7354-7371):}
Alternative ratios excluded:
\begin{itemize}
\item Using $F_1 = 13$ or $F_3 = 89$: Gives $m_p/m_e = 3159$ or $8019$, off by factors 2-4.
\item Using $\chi d^3$ instead of $\chi^2 d^3$: Gives $m_p/m_e = 54$, off by factor 34.
\item Different graph (K₃): Parameters yield $m_p/m_e = 18$, wrong by factor 100.
\end{itemize}

\paragraph{Robustness (Lines 7372-7388):}
Under perturbations:
\begin{itemize}
\item Varying $F_2 \in [16,18]$: Changes $m_p/m_e$ by $\pm 5\%$, still O(1800).
\item Using $F_2 = 16.9$: Gives $m_p/m_e = 1831$, error $0.28\%$ (still excellent).
\end{itemize}

\paragraph{CrossConstraints (Lines 7389-7405):}
Mass ratios connect to:
\begin{itemize}
\item Proton-electron mass ratio appears in atomic spectra.
\item Muon lifetime $\tau_\mu \propto m_\mu^{-5}$ depends on muon mass.
\item Tau decays constrain $m_\tau$ through phase space.
\item Neutron-proton mass difference $m_n - m_p = \chi = 2$ (up to units).
\end{itemize}

\subsection{Summary: Uniform Application}

All six theorems satisfy complete four-part validation:

\begin{center}
\begin{tabular}{lcccc}
\toprule
\textbf{Theorem} & \textbf{Consistency} & \textbf{Exclusivity} & \textbf{Robustness} & \textbf{CrossConstraints} \\
\midrule
$K_4$ Genesis & \checkmark & \checkmark & \checkmark & \checkmark \\
Eigenspace & \checkmark & \checkmark & \checkmark & \checkmark \\
Dimension 3+1 & \checkmark & \checkmark & \checkmark & \checkmark \\
Time Asymmetry & \checkmark & \checkmark & \checkmark & \checkmark \\
Alpha $\alpha^{-1}$ & \checkmark & \checkmark & \checkmark & \checkmark \\
Mass Ratios & \checkmark & \checkmark & \checkmark & \checkmark \\
\bottomrule
\end{tabular}
\end{center}

\noindent This is not selective reporting---\textit{every} major FirstDistinction claim undergoes identical validation.

\section{Comparison to Other Validation Methods}
\label{sec:comparison}

How does four-part validation compare to standard scientific validation?

\subsection{Peer Review}

\textbf{Peer review} relies on expert judgment to assess:
\begin{itemize}[leftmargin=*]
\item Correctness of mathematics
\item Plausibility of interpretations
\item Adequacy of evidence
\end{itemize}

\paragraph{Strengths:}
\begin{itemize}[leftmargin=*]
\item Flexible---can evaluate diverse claims
\item Domain-aware---uses field-specific knowledge
\item Holistic---considers context beyond formal proof
\end{itemize}

\paragraph{Limitations:}
\begin{itemize}[leftmargin=*]
\item Reactive---detects problems after submission
\item Variable---depends on reviewer expertise and diligence
\item Informal---no standardized checklist
\end{itemize}

\paragraph{Four-Part Comparison:}
\begin{itemize}[leftmargin=*]
\item \textit{Proactive}: Validation built into proof structure
\item \textit{Uniform}: Same checks for every theorem
\item \textit{Formal}: Implemented as type-checked Agda records
\end{itemize}

\subsection{Statistical Hypothesis Testing}

\textbf{Statistical testing} uses p-values to reject null hypotheses:
$$\text{p-value} = P(\text{data} \mid \text{null hypothesis})$$

\paragraph{Strengths:}
\begin{itemize}[leftmargin=*]
\item Quantitative---provides numerical confidence level
\item Standard---widely understood methodology
\item Conservative---controls false positive rate
\end{itemize}

\paragraph{Limitations:}
\begin{itemize}[leftmargin=*]
\item Tests randomness---does not address deterministic cherry-picking
\item Requires data---FirstDistinction has 0 fitted parameters
\item Ignores mechanism---low p-value does not prove theory correct
\end{itemize}

\paragraph{Four-Part Comparison:}
\begin{itemize}[leftmargin=*]
\item \textit{Deterministic}: Validates logical necessity, not statistical significance
\item \textit{Parameter-free}: Works without training data
\item \textit{Mechanistic}: Proves why result must hold
\end{itemize}

\subsection{Cross-Validation}

\textbf{Cross-validation} splits data into training and test sets to assess generalization:
$$\text{CV error} = \frac{1}{k} \sum_{i=1}^k \text{error}_{\text{test}_i}$$

\paragraph{Strengths:}
\begin{itemize}[leftmargin=*]
\item Detects overfitting---ensures model generalizes
\item Data-efficient---uses all data for both training and testing
\item Predictive---estimates out-of-sample performance
\end{itemize}

\paragraph{Limitations:}
\begin{itemize}[leftmargin=*]
\item Requires abundant data---FirstDistinction predicts from structure, not data
\item Assumes independent samples---physical constants are not repeated measurements
\item Tests generalization, not necessity---low CV error does not prove result must hold
\end{itemize}

\paragraph{Four-Part Comparison:}
\begin{itemize}[leftmargin=*]
\item \textit{Data-independent}: Validates logical structure, not empirical fit
\item \textit{Necessity-focused}: Proves alternatives impossible, not just unlikely
\item \textit{Structural}: Tests mathematical coherence, not predictive accuracy
\end{itemize}

\subsection{Bayesian Model Selection}

\textbf{Bayesian model selection} compares theories via posterior probabilities:
$$P(M \mid D) \propto P(D \mid M) \cdot P(M)$$

\paragraph{Strengths:}
\begin{itemize}[leftmargin=*]
\item Principled---combines evidence with prior beliefs
\item Penalizes complexity---Occam's razor built in
\item Quantitative---provides model probabilities
\end{itemize}

\paragraph{Limitations:}
\begin{itemize}[leftmargin=*]
\item Requires priors---subjective choice affects conclusion
\item Assumes model space---cannot evaluate structures outside prior
\item Comparative---ranks models, does not validate absolute correctness
\end{itemize}

\paragraph{Four-Part Comparison:}
\begin{itemize}[leftmargin=*]
\item \textit{Absolute}: Validates each claim independently, not comparatively
\item \textit{Prior-free}: No subjective probability distributions
\item \textit{Constructive}: Proves result must hold, not just probably holds
\end{itemize}

\subsection{Complementary, Not Competitive}

Four-part validation is not a replacement for these methods but a \textit{complement}:

\begin{itemize}[leftmargin=*]
\item \textbf{Use peer review} to assess physical plausibility and domain-specific context.
\item \textbf{Use statistical testing} when fitting parameters to data.
\item \textbf{Use cross-validation} to validate empirical models.
\item \textbf{Use Bayesian methods} to compare alternative theories.
\item \textbf{Use four-part validation} to assess pure structural theories with zero free parameters.
\end{itemize}

FirstDistinction occupies a unique niche: theories derived from mathematical necessity rather than empirical fitting. For such theories, four-part validation provides the appropriate rigor.

\section{Implications for Physical Theory}
\label{sec:implications}

The four-part pattern has broader implications beyond FirstDistinction.

\subsection{Template for Theory Construction}

The pattern provides a template for building physical theories from mathematical structure:

\begin{methodbox}
\textbf{Theory Construction Protocol:}
\begin{enumerate}
\item Identify minimal mathematical structure (e.g., $K_4$).
\item Derive consequences through multiple paths (Consistency).
\item Prove alternatives impossible (Exclusivity).
\item Verify stability under perturbations (Robustness).
\item Connect to other validated results (CrossConstraints).
\end{enumerate}
\end{methodbox}

\noindent This protocol is \textit{domain-independent}---applicable to any field attempting to ground physics in pure mathematics.

\subsection{Numerology Detection}

The pattern provides clear criteria to distinguish legitimate theories from numerology:

\begin{center}
\begin{tabular}{lcc}
\toprule
\textbf{Criterion} & \textbf{Numerology} & \textbf{Theory (Four-Part)} \\
\midrule
Multiple derivations? & No (single path) & Yes (Consistency) \\
Alternatives excluded? & No (cherry-picked) & Yes (Exclusivity) \\
Stable under variation? & No (fragile) & Yes (Robustness) \\
Interconnected web? & No (isolated fits) & Yes (CrossConstraints) \\
\bottomrule
\end{tabular}
\end{center}

\noindent Any result satisfying all four criteria is, by definition, not numerology.

\subsection{Mechanization of Validation}

All four components are implementable as formal type-checked records:

\begin{verbatim}
record Validation (Result : Type) : Type where
  field
    consistency    : MultiPath Result
    exclusivity    : (r : Result) → (r ≠ Result) → ⊥
    robustness     : Stable Result Perturbations
    crossConstr    : Interconnected Result OtherResults
\end{verbatim}

\noindent This enables \textit{automated validation checking}---tools can verify that every theorem includes all four components.

\subsection{Raising the Bar for Physical Theory}

Traditional physics validates theories through experiment. But for theories predicting known constants (e.g., $\alpha$, $m_p/m_e$), experiment cannot distinguish between:
\begin{itemize}[leftmargin=*]
\item Fundamental derivation
\item Clever curve fitting
\end{itemize}

Four-part validation provides a \textit{pre-empirical} filter: theories must satisfy all four criteria \textit{before} being tested against observation. This raises the bar---reducing the flood of numerological proposals.

\subsection{Limitations}

The four-part pattern is not a panacea:

\begin{itemize}[leftmargin=*]
\item \textbf{Does not prove physical interpretation}: Mathematical rigor does not guarantee theory describes reality. FirstDistinction's physical hypotheses (e.g., "$K_4$ vertices are spacetime dimensions") remain unproven.

\item \textbf{Does not replace experiment}: Even fully validated predictions must be tested against observation. Four-part validation ensures internal coherence, not empirical correctness.

\item \textbf{Does not address all theory virtues}: Simplicity, elegance, explanatory power are important but not captured by the four parts.
\end{itemize}

\noindent Four-part validation is \textit{necessary} for rigorous structural theory but not \textit{sufficient} for complete physical understanding.

\section{Conclusion}

We have presented the four-part proof pattern underlying FirstDistinction: \textit{Consistency}, \textit{Exclusivity}, \textit{Robustness}, and \textit{CrossConstraints}. We proved that:

\begin{enumerate}[leftmargin=*]
\item All four parts are necessary---removing any reopens exploitable gaps (Section~\ref{sec:necessity}).
\item Four parts suffice---additional checks would be redundant (Section~\ref{sec:sufficiency}).
\item Six FirstDistinction theorems satisfy complete validation (Section~\ref{sec:examples}).
\item The pattern complements standard validation methods (Section~\ref{sec:comparison}).
\item The approach has broader implications for theory construction (Section~\ref{sec:implications}).
\end{enumerate}

The four-part pattern transforms FirstDistinction from "interesting numerical coincidences" to "rigorous mathematical theorems." It is not post-hoc rationalization but structural feature: every major claim implements all four components as type-checked Agda records.

This provides an answer to the skeptic: FirstDistinction \textit{cannot} be numerology because numerology, by definition, fails at least one of the four validation criteria. The results are not cherry-picked (Exclusivity proves alternatives impossible), not curve-fit (zero free parameters), not fragile (Robustness shows stability), and not isolated coincidences (CrossConstraints enforce interdependence).

Whether FirstDistinction describes physical reality remains open---that requires experimental validation of novel predictions. But the mathematics is rigorous, the validation is complete, and the methodology is sound.

The four-part pattern offers a template for future work: pure structural theories grounded in dependent type theory, validated through proactive multi-criteria checking, and distinguished from numerology by construction rather than judgment.

\subsection*{Agda Implementation}

All theorems and validation records are implemented in:
\begin{center}
\texttt{FirstDistinction.agda} \\
7,938 lines, \texttt{-{}-safe -{}-without-K}, 0 axioms, 0 postulates
\end{center}

\noindent Available at: \url{https://github.com/de-johannes/FirstDistinction}

\subsection*{Related Papers}

\begin{itemize}[leftmargin=*]
\item FD-01: Genesis of $K_4$ ($K_4$ uniqueness and forcing)
\item FD-02: Alpha (fine structure constant derivation)
\item FD-03: Dimension (3+1 spacetime from eigenvalues)
\item FD-04: Masses (particle mass ratio predictions)
\item FD-05: Time (Minkowski signature from asymmetry)
\end{itemize}

\begin{thebibliography}{9}

\bibitem{agda}
The Agda Team. \textit{Agda Documentation.}
\url{https://agda.readthedocs.io/}

\bibitem{codata}
CODATA. \textit{Fundamental Physical Constants---Complete Listing.}
NIST, 2022. \url{https://physics.nist.gov/constants}

\bibitem{fd-github}
J. Wielsch. \textit{FirstDistinction Repository.}
\url{https://github.com/de-johannes/FirstDistinction}

\end{thebibliography}

\end{document}
