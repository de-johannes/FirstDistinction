\documentclass[11pt,a4paper]{article}

\usepackage[utf8]{inputenc}
\usepackage[T1]{fontenc}
\usepackage{amsmath,amsthm,amssymb}
\usepackage{mathtools}
\usepackage{geometry}
\usepackage{xcolor}
\usepackage{hyperref}
\usepackage{cleveref}
\usepackage{enumitem}
\usepackage{tcolorbox}
\usepackage{booktabs}
\usepackage{graphicx}

\geometry{a4paper, top=2.5cm, left=2.5cm, right=2.5cm, bottom=3cm}

\definecolor{proof-blue}{RGB}{70,130,180}
\definecolor{hypothesis-green}{RGB}{60,140,90}
\definecolor{box-bg}{RGB}{245,248,250}
\definecolor{hyp-bg}{RGB}{245,252,248}

\hypersetup{
  colorlinks=true,
  linkcolor=proof-blue,
  citecolor=proof-blue,
  urlcolor=proof-blue,
  pdftitle={FD-03: Dimension 3+1 from K4}
}

\theoremstyle{plain}
\newtheorem{theorem}{Theorem}[section]
\newtheorem{lemma}[theorem]{Lemma}
\newtheorem{proposition}[theorem]{Proposition}
\newtheorem{corollary}[theorem]{Corollary}

\theoremstyle{definition}
\newtheorem{definition}[theorem]{Definition}
\newtheorem{hypothesis}[theorem]{Hypothesis}

\theoremstyle{remark}
\newtheorem{remark}[theorem]{Remark}

\tcbuselibrary{skins,breakable}

\newtcolorbox{proven}[1][]{
  colback=box-bg, colframe=proof-blue, fonttitle=\bfseries,
  title={Machine-Verified}, breakable, #1
}

\newtcolorbox{conjecture}[1][]{
  colback=hyp-bg, colframe=hypothesis-green, fonttitle=\bfseries,
  title={Physical Hypothesis}, breakable, #1
}

\title{
  \Large\bfseries
  FD-03: The Emergence of 3+1 Dimensions\\[0.3em]
  \large From $K_4$ Spectral Structure to Spacetime
}

\author{
  Johannes Wielsch\\[0.3em]
  \small Independent Researcher\\
  \small\url{https://github.com/de-johannes/FirstDistinction}
}

\date{December 2025}

\begin{document}

\maketitle

\begin{abstract}
We prove that the complete graph $K_4$ has exactly one zero eigenvalue and three degenerate eigenvalues equal to 4. This spectral structure $(0, 4, 4, 4)$ is machine-verified and follows necessarily from $K_4$'s topology. We then propose the physical hypothesis that these eigenvalues correspond to the $(3+1)$-dimensional structure of spacetime: three spatial dimensions from the degenerate eigenspace, and one temporal dimension from the unique zero eigenvalue (global symmetry). The mathematics is proven under \texttt{--safe --without-K} in Agda (7,938 lines). The physical interpretation remains a testable hypothesis.
\end{abstract}

\section{Introduction}

\subsection{From Graph to Geometry}

In FD-01, we proved that $K_4$ emerges necessarily from the concept of distinction. Now we ask: \emph{What structure does $K_4$ impose?}

The answer lies in spectral graph theory: the eigenvalues of $K_4$'s Laplacian encode its symmetries and constraints. We prove:

\begin{equation}
\text{spectrum}(L_{K_4}) = \{0, 4, 4, 4\}
\end{equation}

One zero (connectivity), three fours (degeneracy). This is not a choice---it is forced by $K_4$'s complete structure.

\subsection{The Central Claim}

\begin{enumerate}
  \item \textbf{Mathematics (proven):} $K_4$ has eigenvalues $(0, 4, 4, 4)$
  \item \textbf{Hypothesis (testable):} These eigenvalues correspond to spacetime's $(3+1)$ structure
  \item \textbf{Prediction:} If this hypothesis is correct, dimensionality is not a free parameter
\end{enumerate}

\subsection{Methodology}

All mathematical proofs formalized in Agda:
\begin{itemize}
  \item \texttt{--safe}: Zero axioms, zero postulates
  \item \texttt{--without-K}: No uniqueness of identity proofs
\end{itemize}

Complete source: \url{https://github.com/de-johannes/FirstDistinction}

\section{The Laplacian Matrix}

\subsection{Definition}

\begin{definition}[Laplacian Matrix]
For a graph $G = (V, E)$ with $n$ vertices, the Laplacian is:
\begin{equation}
L = D - A
\end{equation}
where:
\begin{itemize}
  \item $D_{ij} = \deg(v_i) \cdot \delta_{ij}$ (degree matrix)
  \item $A_{ij} = 1$ if $(v_i, v_j) \in E$, else 0 (adjacency matrix)
\end{itemize}
\end{definition}

\subsection{$K_4$ Laplacian}

For $K_4$, every vertex has degree $\deg = 3$, and every pair is connected:

\begin{equation}
A_{K_4} = \begin{pmatrix}
0 & 1 & 1 & 1 \\
1 & 0 & 1 & 1 \\
1 & 1 & 0 & 1 \\
1 & 1 & 1 & 0
\end{pmatrix}, \quad
D_{K_4} = \begin{pmatrix}
3 & 0 & 0 & 0 \\
0 & 3 & 0 & 0 \\
0 & 0 & 3 & 0 \\
0 & 0 & 0 & 3
\end{pmatrix}
\end{equation}

Therefore:
\begin{equation}
L_{K_4} = D_{K_4} - A_{K_4} = \begin{pmatrix}
3 & -1 & -1 & -1 \\
-1 & 3 & -1 & -1 \\
-1 & -1 & 3 & -1 \\
-1 & -1 & -1 & 3
\end{pmatrix}
\end{equation}

\begin{remark}
This matrix is:
\begin{itemize}
  \item Symmetric: $L^\top = L$
  \item Positive semi-definite: $\mathbf{x}^\top L \mathbf{x} \geq 0$ for all $\mathbf{x}$
  \item Row sums zero: $\sum_j L_{ij} = 0$ (implies zero eigenvalue)
\end{itemize}
\end{remark}

\section{The Eigenvalue Problem}

\subsection{Characteristic Equation}

To find eigenvalues, solve:
\begin{equation}
\det(L_{K_4} - \lambda I) = 0
\end{equation}

Expanding:
\begin{equation}
\det\begin{pmatrix}
3-\lambda & -1 & -1 & -1 \\
-1 & 3-\lambda & -1 & -1 \\
-1 & -1 & 3-\lambda & -1 \\
-1 & -1 & -1 & 3-\lambda
\end{pmatrix} = 0
\end{equation}

\subsection{Symmetry Exploitation}

By symmetry, all rows and columns are equivalent. The matrix has the form:
\begin{equation}
L_{K_4} = 4I - J
\end{equation}
where $J$ is the $4 \times 4$ all-ones matrix.

\textbf{Key insight:} $J$ has eigenvalues $\{4, 0, 0, 0\}$ (rank 1, with eigenvector $(1,1,1,1)^\top$).

Therefore, $L_{K_4} = 4I - J$ has eigenvalues:
\begin{equation}
\lambda(L_{K_4}) = 4 - \lambda(J) = \{4-4, 4-0, 4-0, 4-0\} = \{0, 4, 4, 4\}
\end{equation}

\section{Machine-Verified Proof}

\begin{proven}
\begin{theorem}[K4 Spectrum]
\label{thm:spectrum}
The Laplacian matrix of $K_4$ has exactly four eigenvalues:
\begin{equation}
\text{spectrum}(L_{K_4}) = \{0, 4, 4, 4\}
\end{equation}
with multiplicities:
\begin{itemize}
  \item $\lambda_0 = 0$ (multiplicity 1)
  \item $\lambda_1 = \lambda_2 = \lambda_3 = 4$ (multiplicity 3)
\end{itemize}
\end{theorem}
\end{proven}

\begin{proof}[Proof sketch]
\textbf{Step 1 (Zero eigenvalue):} The all-ones vector $\mathbf{v}_0 = (1,1,1,1)^\top$ satisfies:
\begin{equation}
L_{K_4} \mathbf{v}_0 = \begin{pmatrix}
3-1-1-1 \\ -1+3-1-1 \\ -1-1+3-1 \\ -1-1-1+3
\end{pmatrix} = \begin{pmatrix} 0 \\ 0 \\ 0 \\ 0 \end{pmatrix}
\end{equation}
Hence $\lambda_0 = 0$.

\textbf{Step 2 (Degeneracy):} Consider vectors orthogonal to $(1,1,1,1)^\top$, e.g.:
\begin{align}
\mathbf{v}_1 &= (1, -1, 0, 0)^\top \\
\mathbf{v}_2 &= (1, 0, -1, 0)^\top \\
\mathbf{v}_3 &= (1, 0, 0, -1)^\top
\end{align}

Direct computation shows $L_{K_4} \mathbf{v}_i = 4 \mathbf{v}_i$ for $i=1,2,3$.

Example:
\begin{equation}
L_{K_4} \begin{pmatrix} 1 \\ -1 \\ 0 \\ 0 \end{pmatrix} = \begin{pmatrix}
3 \cdot 1 - 1 \cdot (-1) - 1 \cdot 0 - 1 \cdot 0 \\
-1 \cdot 1 + 3 \cdot (-1) - 1 \cdot 0 - 1 \cdot 0 \\
-1 \cdot 1 - 1 \cdot (-1) + 3 \cdot 0 - 1 \cdot 0 \\
-1 \cdot 1 - 1 \cdot (-1) - 1 \cdot 0 + 3 \cdot 0
\end{pmatrix} = \begin{pmatrix} 4 \\ -4 \\ 0 \\ 0 \end{pmatrix} = 4 \begin{pmatrix} 1 \\ -1 \\ 0 \\ 0 \end{pmatrix}
\end{equation}

\textbf{Step 3 (Completeness):} The four eigenvalues account for the $4 \times 4$ matrix. The subspace of eigenvectors with $\lambda=4$ has dimension 3.

Full proof: lines 2476--2540 of \texttt{FirstDistinction.agda}.
\end{proof}

\subsection{Uniqueness of Zero}

\begin{proven}
\begin{corollary}[Unique Zero Eigenvalue]
\label{cor:unique-zero}
The zero eigenvalue has multiplicity exactly 1, corresponding to the connected nature of $K_4$.
\end{corollary}
\end{proven}

\begin{proof}
The nullspace of $L_{K_4}$ is spanned by the all-ones vector. This reflects global connectivity: all vertices are reachable from any other. If $K_4$ had multiple disconnected components, the zero eigenvalue would have higher multiplicity. Lines 2541--2570.
\end{proof}

\subsection{Degeneracy of Four}

\begin{proven}
\begin{corollary}[Threefold Degeneracy]
\label{cor:degeneracy}
The eigenvalue $\lambda = 4$ has multiplicity exactly 3.
\end{corollary}
\end{proven}

\begin{proof}
The eigenspace for $\lambda = 4$ consists of vectors orthogonal to $(1,1,1,1)^\top$. This is a 3-dimensional subspace of $\mathbb{R}^4$. The three eigenvectors are linearly independent and span the orthogonal complement of the connectivity eigenvector. Lines 2571--2610.
\end{proof}

\section{Graph-Theoretic Interpretation}

\subsection{The Zero Eigenvalue}

\begin{definition}[Algebraic Connectivity]
The second-smallest eigenvalue of a Laplacian is called the \emph{algebraic connectivity} or \emph{Fiedler value}.
\end{definition}

For $K_4$, the Fiedler value is $\lambda_1 = 4$ (not zero). This indicates:
\begin{itemize}
  \item $K_4$ is maximally connected (complete graph)
  \item Removing any edge still leaves the graph connected
  \item The gap between $\lambda_0 = 0$ and $\lambda_1 = 4$ is maximal for 4 vertices
\end{itemize}

\subsection{The Degenerate Eigenspace}

The 3-dimensional eigenspace for $\lambda = 4$ reflects:
\begin{itemize}
  \item Three independent constraints (degrees of freedom)
  \item Orthogonal directions in the space of perturbations
  \item Equal resistance to perturbations in all three directions (isotropy)
\end{itemize}

\section{The Physical Hypothesis}

\subsection{Mapping to Spacetime}

\begin{conjecture}
\begin{hypothesis}[Dimension Correspondence]\label{hyp:dimension}
The eigenvalue structure $(0, 4, 4, 4)$ of $K_4$ corresponds to the $(3+1)$-dimensional structure of spacetime:
\begin{itemize}
  \item \textbf{Zero eigenvalue (multiplicity 1):} Temporal dimension (global connectivity, breaking of symmetry)
  \item \textbf{Degenerate eigenvalue 4 (multiplicity 3):} Three spatial dimensions (isotropy, symmetry)
\end{itemize}
\end{hypothesis}
\end{conjecture}

\subsection{Rationale}

\textbf{Why zero $\to$ time?}
\begin{itemize}
  \item The zero eigenvalue corresponds to the eigenvector $(1,1,1,1)^\top$: all vertices treated equally
  \item This represents \emph{global symmetry}---a distinguished direction
  \item Time is the dimension in which irreversibility and asymmetry manifest (see FD-05: Time from Asymmetry)
\end{itemize}

\textbf{Why $4 = 4 = 4 \to$ space?}
\begin{itemize}
  \item Threefold degeneracy implies three \emph{equivalent} directions
  \item Space is isotropic: no preferred spatial direction
  \item The eigenvalue 4 reflects the constraint imposed by complete connectivity
\end{itemize}

\subsection{Predictions}

If Hypothesis \ref{hyp:dimension} is correct:
\begin{enumerate}
  \item Spatial dimensionality ($d=3$) is not a free parameter---it is forced by $K_4$
  \item Temporal dimensionality ($d=1$) reflects the unique zero eigenvalue
  \item Any theory requiring $d \neq 3$ spatial dimensions contradicts the $K_4$ structure
  \item Higher-dimensional theories (e.g., $d=10$ in string theory) would require justification beyond $K_4$
\end{enumerate}

\section{Alternative Graphs}

\subsection{Why Not $K_3$?}

The spectrum of $K_3$:
\begin{equation}
\text{spectrum}(L_{K_3}) = \{0, 3, 3\}
\end{equation}

This gives $(1+2)$ dimensions---not $(3+1)$. Moreover, $K_3$ fails to achieve closure (FD-01, Theorem 7.1).

\subsection{Why Not $K_5$?}

The spectrum of $K_5$:
\begin{equation}
\text{spectrum}(L_{K_5}) = \{0, 5, 5, 5, 5\}
\end{equation}

This gives $(1+4)$ dimensions. However, $K_5$ is not forced by the genesis mechanism (FD-01, Theorem 7.2).

\subsection{Uniqueness of $K_4$}

\begin{proven}
\begin{theorem}[K4 Dimensional Uniqueness]
\label{thm:dim-unique}
Among complete graphs $K_n$:
\begin{itemize}
  \item Only $K_4$ yields exactly 3 degenerate non-zero eigenvalues
  \item Only $K_4$ satisfies both closure (FD-01) and $(3+1)$ spectrum
\end{itemize}
\end{theorem}
\end{proven}

\begin{proof}
The spectrum of $K_n$ is $\{0, n, n, \ldots, n\}$ with $n$ appearing $(n-1)$ times. For $(3+1)$ structure, we need $n-1 = 3$, hence $n = 4$. Lines 2650--2700.
\end{proof}

\section{Connection to Physics}

\subsection{Dimensionality Problem}

In standard physics, the dimensionality of spacetime is an \emph{input}:
\begin{itemize}
  \item General relativity: $(3+1)$ is assumed
  \item String theory: $10$ or $11$ dimensions postulated
  \item Loop quantum gravity: $(3+1)$ dimensions assumed
\end{itemize}

The FD approach proposes: $(3+1)$ is \emph{derived}, not assumed.

\subsection{Kaluza-Klein and Compactification}

Higher-dimensional theories often invoke \emph{compactification}: extra dimensions are "rolled up" and unobservable. In the FD framework:
\begin{itemize}
  \item $K_4$ provides exactly $(3+1)$ eigenvalues
  \item No extra dimensions exist to compactify
  \item The question shifts from "why don't we see extra dimensions?" to "why would extra dimensions exist?"
\end{itemize}

\subsection{Isotropy of Space}

The threefold degeneracy ($\lambda = 4 = 4 = 4$) implies:
\begin{itemize}
  \item No preferred spatial direction (Copernican principle)
  \item Rotational symmetry (SO(3) naturally emerges)
  \item Equal expansion/contraction in cosmology (if $K_4$ structure is preserved)
\end{itemize}

\section{Validation via Four-Part Structure}

\begin{proven}
\begin{theorem}[Dimension Four-Part Validation]
\label{thm:fourpart-dim}
The $(3+1)$ dimensional structure satisfies:
\begin{enumerate}
  \item \textbf{Consistency:} Eigenvalue count matches graph size ($4 = 1 + 3$)
  \item \textbf{Exclusivity:} Only $K_4$ gives $(0, n, n, n)$ with $n-1=3$
  \item \textbf{Robustness:} Perturbations preserve eigenvalue structure (symmetric matrix properties)
  \item \textbf{Cross-Constraints:} Eigenvalue sum equals trace: $0 + 4 + 4 + 4 = 12 = 4 \cdot \deg = \text{tr}(L_{K_4})$
\end{enumerate}
\end{theorem}
\end{proven}

\begin{proof}
\begin{itemize}
  \item \textbf{Consistency:} $\dim(\ker L) + \dim(\text{eigenspace}_4) = 1 + 3 = 4$
  \item \textbf{Exclusivity:} Proven by checking $K_3, K_4, K_5$ spectra
  \item \textbf{Robustness:} Symmetric matrices have real eigenvalues; perturbations don't destroy degeneracy pattern
  \item \textbf{Cross-Constraints:} $\sum \lambda_i = \text{tr}(L) = \sum \deg(v_i) = 4 \cdot 3 = 12$ \checkmark
\end{itemize}
Lines 2700--2750.
\end{proof}

\section{Experimental Tests}

\subsection{What Would Falsify This Hypothesis?}

\begin{enumerate}
  \item \textbf{Discovery of a fourth spatial dimension:} If experiments reveal $d=4$ spatial dimensions, Hypothesis \ref{hyp:dimension} fails
  \item \textbf{Multiple time dimensions:} If $d_{\text{time}} > 1$, the unique zero eigenvalue cannot explain it
  \item \textbf{Variable dimensionality:} If spacetime dimensionality varies with energy scale or location, $K_4$ structure is insufficient
\end{enumerate}

\subsection{Supportive Evidence}

\begin{itemize}
  \item All experiments confirm $d=3$ spatial dimensions (no deviations at any scale)
  \item Time is observably unique and asymmetric (consistent with unique $\lambda=0$)
  \item Isotropy of space matches degeneracy of $\lambda=4$
\end{itemize}

\subsection{Open Questions}

\begin{itemize}
  \item Can Lorentz signature $(-,+,+,+)$ be derived from eigenvalue signs?
  \item Does the eigenvalue 4 encode physical constants (e.g., coupling strengths)?
  \item Can quantum mechanics emerge from eigenspace structure?
\end{itemize}

\section{Implications}

\subsection{What Is Proven}

\begin{enumerate}
  \item $K_4$ has eigenvalues $(0, 4, 4, 4)$ (machine-verified, zero axioms)
  \item This spectral structure is unique among forced complete graphs
  \item The degeneracy pattern is $1+3$
\end{enumerate}

\subsection{What Is Hypothesized}

\begin{enumerate}
  \item The eigenvalue structure corresponds to spacetime dimensions
  \item Zero eigenvalue $\leftrightarrow$ time (global symmetry)
  \item Three degenerate eigenvalues $\leftrightarrow$ three spatial dimensions (isotropy)
\end{enumerate}

\subsection{Philosophical Implications}

If accepted, this result suggests:
\begin{itemize}
  \item Dimensionality is not arbitrary---it follows from $K_4$
  \item The number 3 (spatial dimensions) is logically necessary
  \item Higher-dimensional theories require justification beyond minimal structure
\end{itemize}

\section{Related Work}

\begin{itemize}
  \item \textbf{Spectral graph theory:} Chung (1997), Mohar (1991)
  \item \textbf{Emergent spacetime:} Hořava-Lifshitz gravity, loop quantum gravity, causal sets
  \item \textbf{Dimension from dynamics:} Carlip (2017)---dimensional reduction at small scales
  \item \textbf{String theory:} Postulates 10 or 11 dimensions, compactifies to 4
\end{itemize}

Our contribution: derivation of $(3+1)$ from $K_4$ spectral structure, with zero free parameters.

\section{Verification}

\subsection{How to Verify}

\begin{verbatim}
git clone https://github.com/de-johannes/FirstDistinction.git
cd FirstDistinction
agda --safe --without-K FirstDistinction.agda
\end{verbatim}

Check lines 2476--2750 for eigenvalue proofs.

\subsection{Proof Statistics}

\begin{center}
\begin{tabular}{lr}
\toprule
\textbf{Metric} & \textbf{Value} \\
\midrule
Total lines & 7,938 \\
Laplacian construction & Lines 2420--2475 \\
Eigenvalue proofs & Lines 2476--2540 \\
Degeneracy analysis & Lines 2571--2610 \\
Uniqueness & Lines 2650--2700 \\
Axioms & 0 \\
Postulates & 0 \\
\bottomrule
\end{tabular}
\end{center}

\section{Conclusion}

We have proven that $K_4$'s Laplacian has spectrum $(0, 4, 4, 4)$: one zero, three fours. This is not a choice---it is forced by complete connectivity.

We hypothesize that this eigenvalue structure corresponds to spacetime's $(3+1)$ dimensions:
\begin{itemize}
  \item One time dimension (unique zero eigenvalue, global symmetry)
  \item Three spatial dimensions (degenerate eigenvalue, isotropy)
\end{itemize}

If correct, dimensionality is not a free parameter. It is derived from the minimal structure forced by distinction itself.

The mathematics is proven. The physics remains to be tested.

\section*{Acknowledgments}

This work benefited from AI assistance (Claude, ChatGPT, DeepSeek, Perplexity) for proof structuring and LaTeX formatting. All mathematical content is the author's responsibility.

\begin{thebibliography}{99}

\bibitem{agda}
The Agda Team. \emph{Agda Documentation}.
\url{https://agda.readthedocs.io/}

\bibitem{chung1997}
F. R. K. Chung. \emph{Spectral Graph Theory}. American Mathematical Society, 1997.

\bibitem{mohar1991}
B. Mohar. \emph{The Laplacian spectrum of graphs}. Graph Theory, Combinatorics, and Applications, 1991.

\bibitem{carlip2017}
S. Carlip. \emph{Dimension and Dimensional Reduction in Quantum Gravity}. Classical and Quantum Gravity, 34(19), 2017.

\bibitem{polchinski1998}
J. Polchinski. \emph{String Theory}, Volume 1. Cambridge University Press, 1998.

\end{thebibliography}

\end{document}
