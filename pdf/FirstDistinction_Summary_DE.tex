\documentclass[11pt,a4paper]{article}

% ============================================================================
% PACKAGES
% ============================================================================

\usepackage[utf8]{inputenc}
\usepackage[T1]{fontenc}
\usepackage[ngerman]{babel}
\usepackage{amsmath,amsthm,amssymb}
\usepackage{mathtools}
\usepackage{geometry}
\usepackage{xcolor}
\usepackage{hyperref}
\usepackage{cleveref}
\usepackage{enumitem}
\usepackage{tcolorbox}
\usepackage{listings}
\usepackage{fancyhdr}
\usepackage{titlesec}
\usepackage{abstract}
\usepackage{booktabs}

% ============================================================================
% GEOMETRY
% ============================================================================

\geometry{
  a4paper,
  left=3cm,
  right=3cm,
  top=3cm,
  bottom=3.5cm
}

% ============================================================================
% COLORS (Sanft, inspiriert von TheIrrefutable)
% ============================================================================

\definecolor{fd-blue}{RGB}{70,130,180}       % Stahlblau
\definecolor{fd-dark}{RGB}{50,50,60}         % Dunkles Grau
\definecolor{fd-light}{RGB}{245,248,250}     % Sehr helles Blaugrau
\definecolor{fd-green}{RGB}{60,120,60}       % Beobachtungsgrün
\definecolor{fd-accent}{RGB}{150,80,80}      % Gedämpftes Rot
\definecolor{code-bg}{RGB}{250,250,252}      % Fast weiß

% ============================================================================
% HYPERREF
% ============================================================================

\hypersetup{
  colorlinks=true,
  linkcolor=fd-blue,
  citecolor=fd-blue,
  urlcolor=fd-blue,
  pdfauthor={Johannes Wielsch},
  pdftitle={First Distinction: Graphentheorie aus einem primitiven Prinzip}
}

% ============================================================================
% SECTION STYLING
% ============================================================================

\titleformat{\section}
  {\Large\bfseries\color{fd-dark}}
  {\thesection}{1em}{}
\titleformat{\subsection}
  {\large\bfseries\color{fd-dark}}
  {\thesubsection}{1em}{}
\titleformat{\subsubsection}
  {\normalsize\bfseries\color{fd-dark}}
  {\thesubsubsection}{1em}{}

% ============================================================================
% THEOREM ENVIRONMENTS
% ============================================================================

\theoremstyle{plain}
\newtheorem{theorem}{Theorem}[section]
\newtheorem{lemma}[theorem]{Lemma}
\newtheorem{proposition}[theorem]{Proposition}
\newtheorem{korollar}[theorem]{Korollar}

\theoremstyle{definition}
\newtheorem{definition}[theorem]{Definition}

\theoremstyle{remark}
\newtheorem{bemerkung}[theorem]{Bemerkung}

% ============================================================================
% CUSTOM BOXES
% ============================================================================

\tcbuselibrary{skins,breakable}

% Bewiesene Resultate (blau)
\newtcolorbox{bewiesen}[1][]{
  colback=fd-light,
  colframe=fd-blue,
  fonttitle=\bfseries,
  title={Bewiesen (Maschinenverifiziert)},
  breakable,
  boxrule=1pt,
  #1
}

% Beobachtungen (grün) - explizit NICHT bewiesene Physik
\newtcolorbox{beobachtung}[1][]{
  colback=green!3,
  colframe=fd-green,
  fonttitle=\bfseries,
  title={Beobachtung (Numerischer Zufall?)},
  breakable,
  boxrule=1pt,
  #1
}

% Schlüsseleinsicht (neutral)
\newtcolorbox{einsicht}[1][]{
  colback=fd-light,
  colframe=fd-dark,
  fonttitle=\bfseries,
  title={Einsicht},
  breakable,
  boxrule=1pt,
  #1
}

% ============================================================================
% CODE LISTINGS
% ============================================================================

\lstset{
  basicstyle=\small\ttfamily,
  backgroundcolor=\color{code-bg},
  frame=single,
  framexleftmargin=5pt,
  framexrightmargin=5pt,
  xleftmargin=10pt,
  xrightmargin=10pt,
  breaklines=true,
  columns=flexible,
  rulecolor=\color{fd-blue!30}
}

% ============================================================================
% HEADER/FOOTER
% ============================================================================

\pagestyle{fancy}
\fancyhf{}
\fancyhead[L]{\small\itshape First Distinction}
\fancyhead[R]{\small\thepage}
\renewcommand{\headrulewidth}{0.4pt}

% ============================================================================
% TITLE
% ============================================================================

\title{
  \Huge\bfseries\color{fd-dark}
  First Distinction\\[0.5em]
  \Large Graphentheorie aus einem primitiven Prinzip\\[1em]
  \large\normalfont Maschinenverifiziert in Agda mit \texttt{--safe --without-K}
}

\author{
  \Large Johannes Wielsch\\[0.5em]
  \normalsize Mit KI-Unterstützung:\\
  \small Claude, ChatGPT, Deepseek, Perplexity
}

\date{
  \today\\[1em]
  \normalsize\href{https://doi.org/10.5281/zenodo.17826218}{DOI: 10.5281/zenodo.17826218}
}

% ============================================================================
% DOCUMENT
% ============================================================================

\begin{document}

\maketitle

% ============================================================================
% ABSTRACT
% ============================================================================

\begin{abstract}
\noindent
\textbf{Was dieses Dokument beweist} (Agda \texttt{--safe --without-K}):

Aus der Prämisse "`etwas kann von etwas unterschieden werden"' emergiert der vollständige Graph $K_4$ als einzige stabile Struktur. Dies ist reine Graphentheorie: 4 Knoten, 6 Kanten, Euler-Charakteristik $\chi = 2$.

\medskip
\textbf{Was dieses Dokument beobachtet} (nicht bewiesen---möglicherweise zufällig):

Die $K_4$-Invarianten stimmen numerisch mit bestimmten physikalischen Konstanten überein. Ob dies bedeutsam oder zufällig ist, ist eine offene Frage, die dieses Dokument nicht beantwortet.

\medskip
\textbf{Status:} Die Mathematik ist maschinenverifiziert. Die Physik-Korrespondenz ist \textit{unbewiesene Hypothese}.
\end{abstract}

\tableofcontents
\newpage

% ============================================================================
\section{Einleitung}
% ============================================================================

\subsection{Was dieses Dokument ist}

Dies ist ein Dokument über \textbf{Graphentheorie}. Konkret: Was ist der einfachste Graph, der allein aus dem Konzept der "`Unterscheidung"' emergieren kann?

Die Antwort, konstruktiv in Agda bewiesen: der vollständige Graph $K_4$ (Tetraeder).

\subsection{Was dieses Dokument nicht ist}

Dies ist \textit{kein} Physik-Paper. Wir behaupten nicht, die physikalische Realität herzuleiten. Wir beobachten, dass bestimmte $K_4$-Invarianten zufällig mit physikalischen Konstanten übereinstimmen. Ob dies bedeutsam oder zufällig ist, ist eine offene Frage.

\subsection{Epistemologische Ehrlichkeit}

In diesem Dokument trennen wir strikt zwischen dem, was \textit{bewiesen} ist, und dem, was \textit{beobachtet} wird:

\begin{bewiesen}
Aussagen in blauen Boxen sind \textbf{mathematisch bewiesen}. Sie kompilieren in Agda unter \texttt{--safe --without-K}. Es sind Theoreme der konstruktiven Mathematik.
\end{bewiesen}

\begin{beobachtung}
Aussagen in grünen Boxen sind \textbf{numerische Beobachtungen}. Sie notieren, dass berechnete Werte zufällig mit experimentellen Messungen übereinstimmen. \textit{Kein kausaler Zusammenhang ist bewiesen.}
\end{beobachtung}

\subsection{Warum Agda?}

\textbf{Agda} ist eine dependently-typed Programmiersprache und Beweisassistent. Anders als Beweise auf Papier wird jeder Schritt maschinell geprüft.

Wir verwenden die striktesten Einstellungen:
\begin{itemize}
  \item \texttt{--safe}: Keine Axiome, keine Postulate, keine Hintertüren
  \item \texttt{--without-K}: Gewährleistet Kompatibilität mit Homotopie-Typentheorie
  \item Keine Bibliotheksimporte: Vollständig eigenständig
\end{itemize}

\textbf{Konsequenz:} Wenn es kompiliert, ist es bewiesen. Keine versteckten Annahmen möglich.

\subsection{Struktur dieses Dokuments}

\begin{enumerate}
  \item \textbf{Teil I: Die Mathematik} --- Was bewiesen ist
  \item \textbf{Teil II: Die Beobachtungen} --- Was bemerkt wird (aber nicht bewiesen)
  \item \textbf{Teil III: Diskussion} --- Offene Fragen und Ehrlichkeit
\end{enumerate}

% ============================================================================
\part{Die Mathematik (Bewiesen)}
% ============================================================================

% ============================================================================
\section{Der Ausgangspunkt: Unterscheidung}
\label{sec:unterscheidung}
% ============================================================================

\subsection{Das primitive Konzept}

\begin{definition}[Unterscheidung]
Eine \textbf{Unterscheidung} ist der primitive Begriff, dass "`dies"' sich von "`jenem"' unterscheidet. In der Typentheorie: ein bewohnter Typ mit entscheidbarer Gleichheit.
\end{definition}

Wir bezeichnen die erste Unterscheidung als $D_0$. Dies ist kein Axiom---es ist die Anerkennung, dass überhaupt etwas zu sagen die Fähigkeit zur Unterscheidung voraussetzt.

\begin{einsicht}
Um zu leugnen, dass Unterscheidung existiert, müssen Sie Ihre Leugnung von ihrem Gegenteil unterscheiden. Das Konzept setzt sich selbst voraus.
\end{einsicht}

\subsection{Formalisierung in Agda}

\begin{lstlisting}[language=Haskell,caption={Der Unterscheidungs-Typ}]
-- Der Einheitstyp: genau ein Bewohner
data Top : Set where
  tt : Top

-- Dies ist D0: die Tatsache, dass "etwas" existiert
D0 : Top
D0 = tt
\end{lstlisting}

Der Einheitstyp $\top$ mit seinem einzigen Bewohner \texttt{tt} \textit{ist} die Formalisierung von "`etwas existiert, das unterschieden werden kann"'.

% ============================================================================
\section{Erzwungene Emergenz: Von \texorpdfstring{$D_0$}{D0} zu \texorpdfstring{$K_4$}{K4}}
\label{sec:emergenz}
% ============================================================================

\subsection{Die Genesis-Kette}

\begin{bewiesen}
\begin{theorem}[Genesis-Kette]
Ausgehend von $D_0$ werden weitere Unterscheidungen erzwungen:
\begin{align*}
D_0 &\Rightarrow D_1 \quad \text{(um $D_0$ von "`nicht-$D_0$"' zu unterscheiden)} \\
D_0, D_1 &\Rightarrow D_2 \quad \text{(um ihren Unterschied zu bezeugen)} \\
D_0, D_1, D_2 &\Rightarrow D_3 \quad \text{(für Abschluss)}
\end{align*}
Bei $n = 4$ schließt sich das System: Jedes Paar $(D_i, D_j)$ hat einen Zeugen unter den verbleibenden zwei.
\end{theorem}
\end{bewiesen}

\textbf{Warum stoppt dies bei 4?}

Mit 4 Unterscheidungen $\{D_0, D_1, D_2, D_3\}$ gibt es $\binom{4}{2} = 6$ Paare. Jedes Paar $(D_i, D_j)$ kann von den anderen beiden Unterscheidungen bezeugt werden. Keine neue Unterscheidung wird erzwungen.

\subsection{Speichersättigung}

Definiere \textbf{Speicher} als die Anzahl unterscheidbarer Paare:
\[
\text{memory}(n) = \binom{n}{2} = \frac{n(n-1)}{2}
\]

\begin{bewiesen}
\begin{theorem}[Speichersättigung]
\begin{align*}
\text{memory}(2) &= 1 \\
\text{memory}(3) &= 3 \\
\text{memory}(4) &= 6 = E(K_4)
\end{align*}
Bei $n=4$ entspricht der Speicher der Kantenanzahl von $K_4$. Der vollständige Graph emergiert.
\end{theorem}
\end{bewiesen}

\subsection{Der vollständige Graph \texorpdfstring{$K_4$}{K4}}

Der vollständige Graph $K_4$ ist der Graph, in dem jeder Knoten mit jedem anderen verbunden ist.

\begin{bewiesen}
\textbf{$K_4$-Invarianten:}
\begin{align*}
V &= 4 && \text{(Knoten)} \\
E &= 6 && \text{(Kanten)} \\
F &= 4 && \text{(Flächen, als Tetraeder)} \\
\chi &= V - E + F = 2 && \text{(Euler-Charakteristik)} \\
\text{deg} &= 3 && \text{(Grad jedes Knotens)}
\end{align*}
\end{bewiesen}

% ============================================================================
\section{\texorpdfstring{$K_4$}{K4}-Eindeutigkeit}
\label{sec:eindeutigkeit}
% ============================================================================

\begin{bewiesen}
\begin{theorem}[$K_4$-Eindeutigkeit]
$K_4$ ist der \textbf{einzige} vollständige Graph, der erfüllt:
\begin{enumerate}
  \item Speichersättigung (Kanten = Paare)
  \item Uniformer Knotengrad (Symmetrie)
  \item Abschluss (jedes Paar hat einen Zeugen)
  \item Minimale Kardinalität für diese Eigenschaften
\end{enumerate}
\end{theorem}
\end{bewiesen}

\textbf{Warum nicht $K_3$?} Drei Knoten ergeben nur 3 Paare, aber keinen Abschluss---jedem Paar fehlt ein Dritt-Zeuge.

\textbf{Warum nicht $K_5$?} Fünf Knoten würden funktionieren, aber $K_4$ ist \textit{minimal}. Der Genesis-Prozess stoppt, sobald Abschluss erreicht ist.

% ============================================================================
\section{Spektrale Eigenschaften von \texorpdfstring{$K_4$}{K4}}
\label{sec:spektral}
% ============================================================================

\subsection{Die Graph-Laplace-Matrix}

Für jeden Graphen $G$ ist die \textbf{Laplace-Matrix}:
\[
L = D - A
\]
wobei $D$ die Gradmatrix und $A$ die Adjazenzmatrix ist.

Für $K_4$:
\[
L_{K_4} = \begin{pmatrix}
3 & -1 & -1 & -1 \\
-1 & 3 & -1 & -1 \\
-1 & -1 & 3 & -1 \\
-1 & -1 & -1 & 3
\end{pmatrix}
\]

\begin{bewiesen}
\begin{theorem}[$K_4$-Laplace-Eigenwerte]
Die Eigenwerte von $L_{K_4}$ sind:
\[
\text{spec}(L_{K_4}) = \{0, 4, 4, 4\}
\]
mit Multiplizitäten $(1, 3)$.
\end{theorem}
\end{bewiesen}

\subsection{Eigenraumdimension}

\begin{bewiesen}
\begin{theorem}[Eigenraumdimension]
Der nicht-triviale Eigenraum von $L_{K_4}$ hat Dimension:
\[
\dim(\ker(L - 4I)) = 3
\]
\end{theorem}
\end{bewiesen}

Dies ist ein Theorem über Graphentheorie. Die Zahl 3 ist eine Invariante von $K_4$.

\subsection{Eigenvektoren}

Die drei Eigenvektoren zum Eigenwert $\lambda = 4$ sind orthonormal und spannen einen 3-dimensionalen Unterraum von $\mathbb{R}^4$ auf.

\begin{bewiesen}
\begin{theorem}[Orthonormalbasis]
Die Eigenvektoren von $L_{K_4}$ für $\lambda = 4$ bilden eine Orthonormalbasis eines 3-dimensionalen Raums:
\begin{align*}
v_1 &= \frac{1}{\sqrt{2}}(1, -1, 0, 0) \\
v_2 &= \frac{1}{\sqrt{6}}(1, 1, -2, 0) \\
v_3 &= \frac{1}{\sqrt{12}}(1, 1, 1, -3)
\end{align*}
\end{theorem}
\end{bewiesen}

% ============================================================================
\section{Kombinatorische Formeln}
\label{sec:formeln}
% ============================================================================

Die $K_4$-Invarianten kombinieren sich zu spezifischen ganzen Zahlen. Dies sind mathematische Fakten, keine Physik.

\subsection{Grundlegende Invarianten}

\begin{bewiesen}
\textbf{$K_4$-Zahlen:}
\begin{align*}
V &= 4 && \text{(Knoten)} \\
E &= 6 && \text{(Kanten)} \\
\chi &= 2 && \text{(Euler-Charakteristik)} \\
\text{deg} &= 3 && \text{(Knotengrad)} \\
\lambda &= 4 && \text{(Spektrallücke)}
\end{align*}
\end{bewiesen}

\subsection{Die Fermat-Verbindung}

$K_4$ hat $V = 4 = 2^2$ Knoten. Dies verbindet sich mit Fermat-Zahlen:
\[
F_n = 2^{2^n} + 1
\]

\begin{bewiesen}
\begin{theorem}[Fermat-Primzahl $F_2$]
\[
F_2 = 2^{2^2} + 1 = 2^4 + 1 = 17
\]
$F_2$ ist prim und erscheint in der $K_4$-Kombinatorik.
\end{theorem}
\end{bewiesen}

\subsection{Abgeleitete Größen}

\begin{bewiesen}
\begin{theorem}[Kombinatorische Formeln aus $K_4$]
\begin{align}
\text{Eigenraumdim} &= V - 1 = 3 \\
2V &= 8 \\
\lambda^3 \cdot \chi + \text{deg}^2 &= 64 \cdot 2 + 9 = 137 \\
\chi^2 \cdot \text{deg}^3 \cdot F_2 &= 4 \cdot 27 \cdot 17 = 1836 \\
\text{deg}^2 \cdot (E + F_2) &= 9 \cdot 23 = 207 \\
F_2 \cdot 207 &= 3519
\end{align}
\end{theorem}
\end{bewiesen}

Dies ist reine Arithmetik. Die Zahlen 137, 1836, 207, 3519 sind mathematische Outputs der $K_4$-Struktur.

\subsection{Die Verschränkungs-Identität}

\begin{bewiesen}
\begin{theorem}[$K_4$-Verschränkungs-Identität]
$K_4$ ist der \textbf{einzige} vollständige Graph, bei dem:
\[
\chi \times \text{deg} = E
\]
Verifikation: $2 \times 3 = 6$ \quad \checkmark
\end{theorem}
\end{bewiesen}

Für $K_3$: $\chi \times \text{deg} = 1 \times 2 = 2 \neq 3 = E$. Scheitert.

Für $K_5$: $\chi \times \text{deg} = 2 \times 4 = 8 \neq 10 = E$. Scheitert.

Nur $K_4$ erfüllt diese Identität.

% ============================================================================
\section{Der Agda-Beweis}
\label{sec:agda}
% ============================================================================

\subsection{Verifikationsstatus}

Alle Behauptungen in Teil I sind formalisiert in:
\begin{itemize}
  \item \texttt{FirstDistinction.agda} (15.000+ Zeilen)
  \item Kompiliert mit \texttt{agda --safe --without-K}
  \item Verfügbar unter: \url{https://github.com/de-johannes/FirstDistinction}
\end{itemize}

\subsection{Wie zu verifizieren}

\begin{lstlisting}[language=bash,caption={Verifikationsbefehl}]
git clone https://github.com/de-johannes/FirstDistinction.git
cd FirstDistinction
agda --safe --without-K FirstDistinction.agda
\end{lstlisting}

Wenn es ohne Fehler kompiliert, sind die Beweise gültig.

\subsection{Was maschinenverifiziert ist}

\begin{bewiesen}
\textbf{Verifizierte Behauptungen:}
\begin{enumerate}
  \item $K_4$ emergiert aus Unterscheidung und Speichersättigung
  \item $K_4$-Eindeutigkeit unter vollständigen Graphen
  \item Laplace-Eigenwerte $\{0, 4, 4, 4\}$
  \item Eigenraumdimension = 3
  \item Alle kombinatorischen Formeln
  \item Verschränkungs-Identität $\chi \times \text{deg} = E$
\end{enumerate}
\end{bewiesen}

% ============================================================================
\part{Die Beobachtungen (Nicht Bewiesen)}
% ============================================================================

Der folgende Abschnitt notiert numerische Zufälle. \textbf{Dies sind keine Theoreme über Physik. Kein kausaler Zusammenhang ist bewiesen.}

% ============================================================================
\section{Dimensions-Zufall}
\label{sec:obs-dim}
% ============================================================================

\begin{beobachtung}
Die $K_4$-Eigenraumdimension ist 3.

Physikalische Beobachtung: Der Raum hat 3 Dimensionen.

\medskip
\textit{Dies könnte sein:}
\begin{itemize}
  \item Zufall
  \item Selektionsbias (wir bemerken Übereinstimmungen, ignorieren Fehlschläge)
  \item Tiefe Verbindung (unbewiesen)
\end{itemize}
\end{beobachtung}

% ============================================================================
\section{Kopplungs-Zufall}
\label{sec:obs-kopplung}
% ============================================================================

\begin{beobachtung}
Die Größe $2V = 2 \times 4 = 8$.

Physikalische Beobachtung: Die Einstein-Feldgleichung verwendet $\kappa = 8\pi G/c^4$, wobei der numerische Faktor 8 ist.

\medskip
\textit{Kein kausaler Zusammenhang bewiesen.}
\end{beobachtung}

% ============================================================================
\section{Feinstruktur-Zufall}
\label{sec:obs-alpha}
% ============================================================================

\begin{beobachtung}
Die Spektralformel ergibt:
\[
\alpha^{-1} = \lambda^3 \cdot \chi + \text{deg}^2 + \frac{V}{\text{deg} \cdot (E^2+1)} = 128 + 9 + \frac{4}{111} = 137{,}036...
\]

Physikalische Beobachtung: Die inverse Feinstrukturkonstante:
\[
\alpha^{-1} = 137{,}035\,999\,177(21)
\]

Übereinstimmung: $\approx 0{,}00003\%$

\medskip
\textbf{Zur $+1$ in $E^2+1$:} Dies ist \textit{kein} willkürlicher Anpassungsparameter. Es folgt dem gleichen \textbf{Ein-Punkt-Kompaktifizierungs}-Muster wie:
\begin{itemize}[nosep]
  \item $V+1=5$ (Ecken + Zentroid)
  \item $2^V+1=17$ (Spinoren + Vakuum)
  \item $E^2+1=37$ (Kanten-Paar-Kopplungen + freier Zustand)
\end{itemize}
Die $+1$ repräsentiert den topologischen Abschluss für den Übergang diskret$\to$kontinuierlich. Siehe \texttt{src/agda/Compactification.agda} für die formale Herleitung.

\medskip
\textit{Obwohl die $+1$ mathematisch abgeleitet ist, ist kein kausaler physikalischer Zusammenhang bewiesen.}
\end{beobachtung}

% ============================================================================
\section{Massenverhältnis-Zufälle}
\label{sec:obs-masse}
% ============================================================================

\begin{beobachtung}
\begin{center}
\begin{tabular}{lccc}
\toprule
\textbf{Formel} & \textbf{$K_4$-Wert} & \textbf{Experiment} & \textbf{Fehler} \\
\midrule
$\chi^2 \cdot \text{deg}^3 \cdot F_2$ & 1836 & 1836,15 & 0,008\% \\
$\text{deg}^2 \cdot (E + F_2)$ & 207 & 206,77 & 0,1\% \\
$F_2 \cdot 207$ & 3519 & 3477 & 1,2\% \\
\bottomrule
\end{tabular}
\end{center}

Physikalische Beobachtung: Diese entsprechen den Proton/Elektron-, Muon/Elektron- und Tau/Elektron-Massenverhältnissen.

\medskip
\textit{Kein kausaler Zusammenhang bewiesen. Könnte Numerologie sein.}
\end{beobachtung}

% ============================================================================
\section{Signatur-Zufall}
\label{sec:obs-signatur}
% ============================================================================

\begin{beobachtung}
$K_4$ hat:
\begin{itemize}
  \item 3 symmetrische Eigenvektoren (räumlich)
  \item 1 asymmetrische Richtung (die Genesis-Sequenz $D_0 \to D_1 \to D_2 \to D_3$ ist irreversibel)
\end{itemize}

Physikalische Beobachtung: Raumzeit hat Signatur $(-,+,+,+)$---eine Zeitdimension, drei Raumdimensionen.

\medskip
\textit{Kein kausaler Zusammenhang bewiesen.}
\end{beobachtung}

% ============================================================================
\section{Zusammenfassung der Beobachtungen}
\label{sec:obs-zusammenfassung}
% ============================================================================

\begin{beobachtung}
\begin{center}
\begin{tabular}{lccl}
\toprule
\textbf{Größe} & \textbf{$K_4$} & \textbf{Physik} & \textbf{Status} \\
\midrule
Raumdimensionen & 3 & 3 & Exakte Übereinstimmung \\
Zeitdimensionen & 1 & 1 & Exakte Übereinstimmung \\
Signatur & $(-,+,+,+)$ & $(-,+,+,+)$ & Exakte Übereinstimmung \\
Kopplungsfaktor & 8 & 8 & Exakte Übereinstimmung \\
$\alpha^{-1}$ & 137,036 & 137,036 & 0,00003\% \\
$m_p/m_e$ & 1836 & 1836,15 & 0,008\% \\
$m_\mu/m_e$ & 207 & 206,77 & 0,1\% \\
$m_\tau/m_e$ & 3519 & 3477 & 1,2\% \\
\bottomrule
\end{tabular}
\end{center}

\textbf{4 exakte Übereinstimmungen. 4 nahe Übereinstimmungen.}

\medskip
\textit{Ist dies bedeutsam oder zufällig? Dieses Dokument beantwortet diese Frage nicht.}
\end{beobachtung}

% ============================================================================
\part{Diskussion}
% ============================================================================

% ============================================================================
\section{Was tatsächlich bewiesen ist}
\label{sec:bewiesen-zusammenfassung}
% ============================================================================

\begin{bewiesen}
\textbf{Mathematische Theoreme (Agda \texttt{--safe}):}
\begin{enumerate}
  \item Aus "`Unterscheidung existiert"' emergiert der vollständige Graph $K_4$ eindeutig
  \item $K_4$ hat spezifische Invarianten: $V=4$, $E=6$, $\chi=2$, $\text{deg}=3$
  \item Die Laplace-Matrix hat Eigenwerte $\{0,4,4,4\}$
  \item Die Eigenraumdimension ist 3
  \item Spezifische kombinatorische Formeln ergeben 137, 1836, 207, 3519
  \item Die Verschränkungs-Identität $\chi \cdot \text{deg} = E$ ist einzigartig für $K_4$
\end{enumerate}

Dies sind \textbf{Fakten über Graphentheorie}, maschinenverifiziert.
\end{bewiesen}

% ============================================================================
\section{Was nicht bewiesen ist}
\label{sec:nicht-bewiesen}
% ============================================================================

\begin{beobachtung}
\textbf{Nicht bewiesen:}
\begin{enumerate}
  \item Dass $K_4$ physikalische Raumzeit "`ist"'
  \item Dass die Zahl 137,036 die Feinstrukturkonstante "`ist"'
  \item Dass 1836 das Proton/Elektron-Massenverhältnis "`ist"'
  \item Dass irgendeine dieser Korrespondenzen bedeutsam ist
  \item Dass die numerischen Übereinstimmungen nicht zufällig sind
\end{enumerate}

Die Behauptung "`Mathematik bestimmt Physik"' ist eine \textbf{philosophische Hypothese}, kein Theorem.
\end{beobachtung}

% ============================================================================
\section{Warum nicht \texorpdfstring{$K_3$}{K3} oder \texorpdfstring{$K_5$}{K5}?}
\label{sec:warum-k4}
% ============================================================================

\begin{bewiesen}
\textbf{Mathematische Antwort:} $K_4$ ist der einzige minimale vollständige Graph, der Speichersättigung und Abschluss erfüllt. Dies ist bewiesen.
\end{bewiesen}

\begin{beobachtung}
\textbf{Numerische Beobachtung:} Wenn wir dieselben Formeln für $K_3$ und $K_5$ berechnen:

\begin{center}
\begin{tabular}{lcccc}
\toprule
\textbf{Größe} & \textbf{$K_3$} & \textbf{$K_4$} & \textbf{$K_5$} & \textbf{Expt.} \\
\midrule
Eigenraumdim & 2 & 3 & 4 & 3 \\
$2V$ & 6 & 8 & 10 & 8 \\
"`$\alpha^{-1}$"' & 31 & 137 & 266 & 137 \\
"`$m_p/m_e$"' & 288 & 1836 & 8448 & 1836 \\
\bottomrule
\end{tabular}
\end{center}

Nur $K_4$-Werte stimmen mit dem Experiment überein.

\medskip
\textit{Dies könnte Evidenz für eine tiefe Verbindung sein, oder es könnte Selektionsbias sein---wir bemerken nur, wenn Formeln passen.}
\end{beobachtung}

% ============================================================================
\section{Mögliche Interpretationen}
\label{sec:interpretationen}
% ============================================================================

\subsection{Interpretation 1: Zufall}

Die Zahlen stimmen zufällig überein. Mit genügend Formeln und genügend Konstanten werden einige zufällig passen. Dies ist die Nullhypothese und kann nicht ausgeschlossen werden.

\subsection{Interpretation 2: Selektionsbias}

Wir haben Formeln gefunden, die zu den Daten passen. Dies ist eine Form von Numerologie. Die Tatsache, dass Formeln existieren, bedeutet nicht, dass sie fundamental sind.

\subsection{Interpretation 3: Tiefe Verbindung}

Mathematik schränkt Physik ein. Die Struktur der Unterscheidung \textit{ist} die Struktur der Realität. Dies ist die stärkste Behauptung und die am wenigsten bewiesene.

\begin{einsicht}
Dieses Dokument präsentiert die Mathematik ehrlich. Die Interpretation bleibt dem Leser überlassen. Wir behaupten nicht, dass Interpretation 3 korrekt ist---nur dass die numerischen Übereinstimmungen bemerkenswert genug sind, um Aufmerksamkeit zu verdienen.
\end{einsicht}

% ============================================================================
\section{Was dies falsifizieren würde}
\label{sec:falsifikation}
% ============================================================================

\begin{enumerate}
  \item \textbf{Mathematischer Fehler:} Finden Sie einen Bug im Agda-Code. Dies würde die Beweise falsifizieren.
  \item \textbf{Alternative Herleitung:} Zeigen Sie, dass ein anderer Graph (nicht $K_4$) aus Unterscheidung emergiert.
  \item \textbf{Versteckter Parameter:} Finden Sie einen einstellbaren Parameter in den Formeln.
  \item \textbf{Bessere Numerologie:} Finden Sie einfachere Formeln, die die Konstanten besser treffen.
\end{enumerate}

Die mathematischen Behauptungen sind falsifizierbar durch das Finden von Fehlern im Agda-Code. Die physikalische Interpretation ist schwerer zu falsifizieren, aber klar als Hypothese gekennzeichnet.

% ============================================================================
\section{Offene Fragen}
\label{sec:offen}
% ============================================================================

\begin{enumerate}
  \item \textbf{Warum Fermat-Primzahlen?} Das Erscheinen von $F_2 = 17$ ist unerklärt.
  \item \textbf{Fraktionale Präzision:} Die Formel ergibt 137,036, das Experiment ergibt 137,035999. Woher kommt die Differenz?
  \item \textbf{Andere Konstanten:} Kann dieser Ansatz Konstanten vorhersagen, die wir noch nicht abgeglichen haben?
  \item \textbf{Dynamik:} $K_4$ ist statisch. Wie emergieren Bewegungsgleichungen?
\end{enumerate}

% ============================================================================
\section{Schlussfolgerung}
\label{sec:schluss}
% ============================================================================

\subsection{Zusammenfassung}

\begin{bewiesen}
\textbf{Was wir bewiesen haben:}
\begin{itemize}
  \item $K_4$ emergiert eindeutig aus dem Konzept der Unterscheidung
  \item $K_4$ hat spezifische spektrale und kombinatorische Eigenschaften
  \item Diese Eigenschaften berechnen sich zu spezifischen Zahlen
\end{itemize}
Alle Beweise sind maschinenverifiziert in Agda.
\end{bewiesen}

\begin{beobachtung}
\textbf{Was wir beobachtet haben:}
\begin{itemize}
  \item Diese Zahlen stimmen mit physikalischen Konstanten überein
  \item Die Übereinstimmungen sind präzise (0,00003\% bis 1,2\%)
  \item Nur $K_4$ produziert passende Werte
\end{itemize}
Kein kausaler Zusammenhang ist bewiesen.
\end{beobachtung}

\subsection{Abschließende Aussage}

\begin{einsicht}
Die Mathematik ist sicher. Die Physik ist Hypothese.

Wir präsentieren die stärkstmögliche mathematische Evidenz, aber wir behaupten nicht, Physik aus reinem Denken hergeleitet zu haben. Diese Behauptung würde einen Beweis erfordern, den wir nicht haben.

Wenn Sie einen Fehler finden, öffnen Sie ein Issue. Wir wollen es wissen.
\end{einsicht}

% ============================================================================
\section{Notationsreferenz}
\label{sec:notation}
% ============================================================================

\begin{tabular}{ll}
\toprule
\textbf{Symbol} & \textbf{Bedeutung} \\
\midrule
$D_0, D_1, D_2, D_3$ & Die vier primordialen Unterscheidungen \\
$K_4$ & Vollständiger Graph auf 4 Knoten \\
$V = 4$ & Knotenanzahl \\
$E = 6$ & Kantenanzahl \\
$F = 4$ & Flächenanzahl (als Tetraeder) \\
$\chi = 2$ & Euler-Charakteristik \\
$\text{deg} = 3$ & Knotengrad \\
$\lambda = 4$ & Nicht-trivialer Laplace-Eigenwert \\
$F_2 = 17$ & Fermat-Primzahl $2^{2^2} + 1$ \\
$L$ & Graph-Laplace-Matrix \\
\bottomrule
\end{tabular}

% ============================================================================
% BIBLIOGRAPHY
% ============================================================================

\begin{thebibliography}{99}

\bibitem{martinlof1972}
P. Martin-Löf, \textit{An Intuitionistic Theory of Types}, 
Twenty-Five Years of Constructive Type Theory (1972).

\bibitem{agda}
The Agda Team, \textit{Agda Documentation}, 
\url{https://agda.readthedocs.io/}

\bibitem{codata2018}
CODATA, \textit{Recommended Values of the Fundamental Physical Constants: 2018}, 
Rev. Mod. Phys. 93, 025010 (2021).

\bibitem{spencerbrown}
G. Spencer-Brown, \textit{Laws of Form}, 
Julian Press (1969).

\end{thebibliography}

\end{document}
