\documentclass[11pt,a4paper]{article}

% ============================================================================
% PACKAGES
% ============================================================================

\usepackage[utf8]{inputenc}
\usepackage[T1]{fontenc}
\usepackage{amsmath,amsthm,amssymb}
\usepackage{mathtools}
\usepackage{geometry}
\usepackage{xcolor}
\usepackage{hyperref}
\usepackage{cleveref}
\usepackage{enumitem}
\usepackage{tcolorbox}
\usepackage{listings}
\usepackage{fancyhdr}
\usepackage{titlesec}
\usepackage{abstract}
\usepackage{booktabs}

% ============================================================================
% GEOMETRY
% ============================================================================

\geometry{
  a4paper,
  left=3cm,
  right=3cm,
  top=3cm,
  bottom=3.5cm
}

% ============================================================================
% COLORS (Soft, inspired by TheIrrefutable)
% ============================================================================

\definecolor{fd-blue}{RGB}{70,130,180}       % Steel blue
\definecolor{fd-dark}{RGB}{50,50,60}         % Dark gray
\definecolor{fd-light}{RGB}{245,248,250}     % Very light blue-gray
\definecolor{fd-green}{RGB}{60,120,60}       % Observation green
\definecolor{fd-accent}{RGB}{150,80,80}      % Muted red
\definecolor{code-bg}{RGB}{250,250,252}      % Almost white

% ============================================================================
% HYPERREF
% ============================================================================

\hypersetup{
  colorlinks=true,
  linkcolor=fd-blue,
  citecolor=fd-blue,
  urlcolor=fd-blue,
  pdfauthor={Johannes Wielsch},
  pdftitle={First Distinction: Graph Theory from a Primitive Principle}
}

% ============================================================================
% SECTION STYLING
% ============================================================================

\titleformat{\section}
  {\Large\bfseries\color{fd-dark}}
  {\thesection}{1em}{}
\titleformat{\subsection}
  {\large\bfseries\color{fd-dark}}
  {\thesubsection}{1em}{}
\titleformat{\subsubsection}
  {\normalsize\bfseries\color{fd-dark}}
  {\thesubsubsection}{1em}{}

% ============================================================================
% THEOREM ENVIRONMENTS
% ============================================================================

\theoremstyle{plain}
\newtheorem{theorem}{Theorem}[section]
\newtheorem{lemma}[theorem]{Lemma}
\newtheorem{proposition}[theorem]{Proposition}
\newtheorem{corollary}[theorem]{Corollary}

\theoremstyle{definition}
\newtheorem{definition}[theorem]{Definition}

\theoremstyle{remark}
\newtheorem{remark}[theorem]{Remark}

% ============================================================================
% CUSTOM BOXES
% ============================================================================

\tcbuselibrary{skins,breakable}

% Proven results (blue)
\newtcolorbox{proven}[1][]{
  colback=fd-light,
  colframe=fd-blue,
  fonttitle=\bfseries,
  title={Proven (Machine-Verified)},
  breakable,
  boxrule=1pt,
  #1
}

% Observations (green) - explicitly NOT proven physics
\newtcolorbox{observation}[1][]{
  colback=green!3,
  colframe=fd-green,
  fonttitle=\bfseries,
  title={Observation (Numerical Coincidence?)},
  breakable,
  boxrule=1pt,
  #1
}

% Key insight (neutral)
\newtcolorbox{keyinsight}[1][]{
  colback=fd-light,
  colframe=fd-dark,
  fonttitle=\bfseries,
  title={Key Insight},
  breakable,
  boxrule=1pt,
  #1
}

% ============================================================================
% CODE LISTINGS
% ============================================================================

\lstset{
  basicstyle=\small\ttfamily,
  backgroundcolor=\color{code-bg},
  frame=single,
  framexleftmargin=5pt,
  framexrightmargin=5pt,
  xleftmargin=10pt,
  xrightmargin=10pt,
  breaklines=true,
  columns=flexible,
  rulecolor=\color{fd-blue!30}
}

% ============================================================================
% HEADER/FOOTER
% ============================================================================

\pagestyle{fancy}
\fancyhf{}
\fancyhead[L]{\small\itshape First Distinction}
\fancyhead[R]{\small\thepage}
\renewcommand{\headrulewidth}{0.4pt}

% ============================================================================
% TITLE
% ============================================================================

\title{
  \Huge\bfseries\color{fd-dark}
  First Distinction\\[0.5em]
  \Large Graph Theory from a Primitive Principle\\[1em]
  \large\normalfont Machine-Verified in Agda with \texttt{--safe --without-K}
}

\author{
  \Large Johannes Wielsch\\[0.5em]
  \normalsize With AI Collaboration:\\
  \small Claude, ChatGPT, Deepseek, Perplexity
}

\date{
  \today\\[1em]
  \normalsize\href{https://doi.org/10.5281/zenodo.17826218}{DOI: 10.5281/zenodo.17826218}
}

% ============================================================================
% DOCUMENT
% ============================================================================

\begin{document}

\maketitle

% ============================================================================
% ABSTRACT
% ============================================================================

\begin{abstract}
\noindent
\textbf{What this document proves} (Agda \texttt{--safe --without-K}):

From the premise ``something can be distinguished from something,'' the complete graph $K_4$ emerges as the unique stable structure. This is pure graph theory: 4 vertices, 6 edges, Euler characteristic $\chi = 2$.

\medskip
\textbf{What this document observes} (not proven---possibly coincidental):

The $K_4$ invariants numerically match certain physical constants with surprising precision. Whether this is meaningful or coincidental is an open question that this document does not answer.

\medskip
\textbf{Status:} The mathematics is machine-verified. The physics correspondence is \textit{unproven hypothesis}.
\end{abstract}

\tableofcontents
\newpage

% ============================================================================
\section{Introduction}
% ============================================================================

\subsection{What This Document Is}

This is a document about \textbf{graph theory}. Specifically: what is the simplest graph that can emerge from the concept of ``distinction'' alone?

The answer, proven constructively in Agda: the complete graph $K_4$ (tetrahedron).

\subsection{What This Document Is Not}

This is \textit{not} a physics paper. We do not claim to derive physical reality. We observe that certain $K_4$ invariants happen to match physical constants with surprising precision. Whether this is meaningful or coincidental is an open question.

\subsection{Epistemological Honesty}

Throughout this document, we maintain strict separation between what is \textit{proven} and what is \textit{observed}:

\begin{proven}
Statements in blue boxes are \textbf{mathematically proven}. They type-check in Agda under \texttt{--safe --without-K}. They are theorems of constructive mathematics.
\end{proven}

\begin{observation}
Statements in green boxes are \textbf{numerical observations}. They note that computed values happen to match experimental measurements. \textit{No causal connection is proven.}
\end{observation}

\subsection{Why Agda?}

\textbf{Agda} is a dependently-typed programming language and proof assistant. Unlike proofs on paper, every step is machine-checked.

We use the strictest settings:
\begin{itemize}
  \item \texttt{--safe}: No axioms, no postulates, no escape hatches
  \item \texttt{--without-K}: Ensures compatibility with Homotopy Type Theory
  \item No library imports: Fully self-contained
\end{itemize}

\textbf{Consequence:} If it compiles, it's proven. No hidden assumptions possible.

\subsection{Structure of This Document}

\begin{enumerate}
  \item \textbf{Part I: The Mathematics} --- What is proven
  \item \textbf{Part II: The Observations} --- What is noticed (but not proven)
  \item \textbf{Part III: Discussion} --- Open questions and honesty
\end{enumerate}

% ============================================================================
\part{The Mathematics (Proven)}
% ============================================================================

% ============================================================================
\section{The Starting Point: Distinction}
\label{sec:distinction}
% ============================================================================

\subsection{The Primitive Concept}

\begin{definition}[Distinction]
A \textbf{distinction} is the primitive notion that ``this'' differs from ``that.'' In type theory: an inhabited type with decidable equality.
\end{definition}

We denote the first distinction as $D_0$. This is not an axiom---it is the recognition that to say anything at all presupposes the ability to distinguish.

\begin{keyinsight}
To deny that distinction exists, you must distinguish your denial from its opposite. The concept is self-presupposing.
\end{keyinsight}

\subsection{Formalization in Agda}

\begin{lstlisting}[language=Haskell,caption={The Distinction Type}]
-- The unit type: exactly one inhabitant
data Top : Set where
  tt : Top

-- This is D0: the fact that "something" exists
D0 : Top
D0 = tt
\end{lstlisting}

The unit type $\top$ with its single inhabitant \texttt{tt} \textit{is} the formalization of ``something exists that can be distinguished.''

% ============================================================================
\section{Forced Emergence: From \texorpdfstring{$D_0$}{D0} to \texorpdfstring{$K_4$}{K4}}
\label{sec:emergence}
% ============================================================================

\subsection{The Genesis Chain}

\begin{proven}
\begin{theorem}[Genesis Chain]
Starting from $D_0$, additional distinctions are forced:
\begin{align*}
D_0 &\Rightarrow D_1 \quad \text{(to distinguish $D_0$ from ``not-$D_0$'')} \\
D_0, D_1 &\Rightarrow D_2 \quad \text{(to witness their difference)} \\
D_0, D_1, D_2 &\Rightarrow D_3 \quad \text{(for closure)}
\end{align*}
At $n = 4$, the system closes: every pair $(D_i, D_j)$ has a witness among the remaining two.
\end{theorem}
\end{proven}

\textbf{Why does this stop at 4?}

With 4 distinctions $\{D_0, D_1, D_2, D_3\}$, there are $\binom{4}{2} = 6$ pairs. Each pair $(D_i, D_j)$ can be witnessed by the other two distinctions. No new distinction is forced.

\subsection{Memory Saturation}

Define \textbf{memory} as the count of distinguishable pairs:
\[
\text{memory}(n) = \binom{n}{2} = \frac{n(n-1)}{2}
\]

\begin{proven}
\begin{theorem}[Memory Saturation]
\begin{align*}
\text{memory}(2) &= 1 \\
\text{memory}(3) &= 3 \\
\text{memory}(4) &= 6 = E(K_4)
\end{align*}
At $n=4$, memory equals the edge count of $K_4$. The complete graph emerges.
\end{theorem}
\end{proven}

\subsection{The Complete Graph \texorpdfstring{$K_4$}{K4}}

The complete graph $K_4$ is the graph where every vertex is connected to every other vertex.

\begin{proven}
\textbf{$K_4$ Invariants:}
\begin{align*}
V &= 4 && \text{(vertices)} \\
E &= 6 && \text{(edges)} \\
F &= 4 && \text{(faces, as tetrahedron)} \\
\chi &= V - E + F = 2 && \text{(Euler characteristic)} \\
\text{deg} &= 3 && \text{(degree of each vertex)}
\end{align*}
\end{proven}

% ============================================================================
\section{\texorpdfstring{$K_4$}{K4} Uniqueness}
\label{sec:uniqueness}
% ============================================================================

\begin{proven}
\begin{theorem}[$K_4$ Uniqueness]
$K_4$ is the \textbf{unique} complete graph satisfying:
\begin{enumerate}
  \item Memory saturation (edges = pairs)
  \item Uniform vertex degree (symmetry)
  \item Closure (every pair has a witness)
  \item Minimal cardinality for these properties
\end{enumerate}
\end{theorem}
\end{proven}

\textbf{Why not $K_3$?} Three vertices give only 3 pairs, but no closure---each pair lacks a third-party witness.

\textbf{Why not $K_5$?} Five vertices would work, but $K_4$ is \textit{minimal}. The genesis process stops as soon as closure is achieved.

% ============================================================================
\section{Spectral Properties of \texorpdfstring{$K_4$}{K4}}
\label{sec:spectral}
% ============================================================================

\subsection{The Graph Laplacian}

For any graph $G$, the \textbf{Laplacian matrix} is:
\[
L = D - A
\]
where $D$ is the degree matrix and $A$ is the adjacency matrix.

For $K_4$:
\[
L_{K_4} = \begin{pmatrix}
3 & -1 & -1 & -1 \\
-1 & 3 & -1 & -1 \\
-1 & -1 & 3 & -1 \\
-1 & -1 & -1 & 3
\end{pmatrix}
\]

\begin{proven}
\begin{theorem}[$K_4$ Laplacian Eigenvalues]
The eigenvalues of $L_{K_4}$ are:
\[
\text{spec}(L_{K_4}) = \{0, 4, 4, 4\}
\]
with multiplicities $(1, 3)$.
\end{theorem}
\end{proven}

\subsection{Eigenspace Dimension}

\begin{proven}
\begin{theorem}[Eigenspace Dimension]
The non-trivial eigenspace of $L_{K_4}$ has dimension:
\[
\dim(\ker(L - 4I)) = 3
\]
\end{theorem}
\end{proven}

This is a theorem about graph theory. The number 3 is an invariant of $K_4$.

\subsection{Eigenvectors}

The three eigenvectors corresponding to $\lambda = 4$ are orthonormal and span a 3-dimensional subspace of $\mathbb{R}^4$. 

\begin{proven}
\begin{theorem}[Orthonormal Basis]
The eigenvectors of $L_{K_4}$ for $\lambda = 4$ form an orthonormal basis for a 3-dimensional space:
\begin{align*}
v_1 &= \frac{1}{\sqrt{2}}(1, -1, 0, 0) \\
v_2 &= \frac{1}{\sqrt{6}}(1, 1, -2, 0) \\
v_3 &= \frac{1}{\sqrt{12}}(1, 1, 1, -3)
\end{align*}
\end{theorem}
\end{proven}

% ============================================================================
\section{Combinatorial Formulas}
\label{sec:formulas}
% ============================================================================

The $K_4$ invariants combine to produce specific integers. These are mathematical facts, not physics.

\subsection{Basic Invariants}

\begin{proven}
\textbf{$K_4$ Numbers:}
\begin{align*}
V &= 4 && \text{(vertices)} \\
E &= 6 && \text{(edges)} \\
\chi &= 2 && \text{(Euler characteristic)} \\
\text{deg} &= 3 && \text{(vertex degree)} \\
\lambda &= 4 && \text{(spectral gap)}
\end{align*}
\end{proven}

\subsection{The Fermat Connection}

$K_4$ has $V = 4 = 2^2$ vertices. This connects to Fermat numbers:
\[
F_n = 2^{2^n} + 1
\]

\begin{proven}
\begin{theorem}[Fermat Prime $F_2$]
\[
F_2 = 2^{2^2} + 1 = 2^4 + 1 = 17
\]
$F_2$ is prime, and appears in $K_4$ combinatorics.
\end{theorem}
\end{proven}

\subsection{Derived Quantities}

\begin{proven}
\begin{theorem}[Combinatorial Formulas from $K_4$]
\begin{align}
\text{Eigenspace dim} &= V - 1 = 3 \\
2V &= 8 \\
\lambda^3 \cdot \chi + \text{deg}^2 &= 64 \cdot 2 + 9 = 137 \\
\chi^2 \cdot \text{deg}^3 \cdot F_2 &= 4 \cdot 27 \cdot 17 = 1836 \\
\text{deg}^2 \cdot (E + F_2) &= 9 \cdot 23 = 207 \\
F_2 \cdot 207 &= 3519
\end{align}
\end{theorem}
\end{proven}

These are pure arithmetic. The numbers 137, 1836, 207, 3519 are mathematical outputs of $K_4$ structure.

\subsection{The Entanglement Identity}

\begin{proven}
\begin{theorem}[$K_4$ Entanglement Identity]
$K_4$ is the \textbf{unique} complete graph where:
\[
\chi \times \text{deg} = E
\]
Verification: $2 \times 3 = 6$ \quad \checkmark
\end{theorem}
\end{proven}

For $K_3$: $\chi \times \text{deg} = 1 \times 2 = 2 \neq 3 = E$. Fails.

For $K_5$: $\chi \times \text{deg} = 2 \times 4 = 8 \neq 10 = E$. Fails.

Only $K_4$ satisfies this identity.

% ============================================================================
\section{The Agda Proof}
\label{sec:agda}
% ============================================================================

\subsection{Verification Status}

All claims in Part I are formalized in:
\begin{itemize}
  \item \texttt{FirstDistinction.agda} (15,000+ lines)
  \item Compiles with \texttt{agda --safe --without-K}
  \item Available at: \url{https://github.com/de-johannes/FirstDistinction}
\end{itemize}

\subsection{How to Verify}

\begin{lstlisting}[language=bash,caption={Verification Command}]
git clone https://github.com/de-johannes/FirstDistinction.git
cd FirstDistinction
agda --safe --without-K FirstDistinction.agda
\end{lstlisting}

If it compiles without errors, the proofs are valid.

\subsection{What Is Machine-Verified}

\begin{proven}
\textbf{Verified claims:}
\begin{enumerate}
  \item $K_4$ emerges from distinction and memory saturation
  \item $K_4$ uniqueness among complete graphs
  \item Laplacian eigenvalues $\{0, 4, 4, 4\}$
  \item Eigenspace dimension = 3
  \item All combinatorial formulas
  \item Entanglement identity $\chi \times \text{deg} = E$
\end{enumerate}
\end{proven}

% ============================================================================
\part{The Observations (Not Proven)}
% ============================================================================

The following section notes numerical coincidences. \textbf{These are not theorems about physics. No causal connection is proven.}

% ============================================================================
\section{Dimensional Coincidence}
\label{sec:obs-dim}
% ============================================================================

\begin{observation}
The $K_4$ eigenspace dimension is 3.

Physical observation: Space has 3 dimensions.

\medskip
\textit{This could be:}
\begin{itemize}
  \item Coincidence
  \item Selection bias (we notice matches, ignore misses)
  \item Deep connection (unproven)
\end{itemize}
\end{observation}

% ============================================================================
\section{Coupling Coincidence}
\label{sec:obs-coupling}
% ============================================================================

\begin{observation}
The quantity $2V = 2 \times 4 = 8$.

Physical observation: The Einstein field equation uses $\kappa = 8\pi G/c^4$, where the numerical factor is 8.

\medskip
\textit{No causal connection proven.}
\end{observation}

% ============================================================================
\section{Fine Structure Coincidence}
\label{sec:obs-alpha}
% ============================================================================

\begin{observation}
The spectral formula yields:
\[
\lambda^3 \cdot \chi + \text{deg}^2 + \text{correction} = 137.036...
\]

Physical observation: The inverse fine structure constant:
\[
\alpha^{-1} = 137.035\,999\,177(21)
\]

Agreement: $\approx 0.00003\%$

\medskip
\textit{No causal connection proven. The ``correction term'' involves $V/(\text{deg} \cdot (E^2+1))$ which could be post-hoc fitting.}
\end{observation}

% ============================================================================
\section{Mass Ratio Coincidences}
\label{sec:obs-mass}
% ============================================================================

\begin{observation}
\begin{center}
\begin{tabular}{lccc}
\toprule
\textbf{Formula} & \textbf{$K_4$ Value} & \textbf{Experimental} & \textbf{Error} \\
\midrule
$\chi^2 \cdot \text{deg}^3 \cdot F_2$ & 1836 & 1836.15 & 0.008\% \\
$\text{deg}^2 \cdot (E + F_2)$ & 207 & 206.77 & 0.1\% \\
$F_2 \cdot 207$ & 3519 & 3477 & 1.2\% \\
\bottomrule
\end{tabular}
\end{center}

Physical observation: These match proton/electron, muon/electron, and tau/electron mass ratios.

\medskip
\textit{No causal connection proven. Could be numerology.}
\end{observation}

% ============================================================================
\section{Signature Coincidence}
\label{sec:obs-signature}
% ============================================================================

\begin{observation}
$K_4$ has:
\begin{itemize}
  \item 3 symmetric eigenvectors (spatial)
  \item 1 asymmetric direction (the genesis sequence $D_0 \to D_1 \to D_2 \to D_3$ is irreversible)
\end{itemize}

Physical observation: Spacetime has signature $(-,+,+,+)$---one time dimension, three space dimensions.

\medskip
\textit{No causal connection proven.}
\end{observation}

% ============================================================================
\section{Summary of Observations}
\label{sec:obs-summary}
% ============================================================================

\begin{observation}
\begin{center}
\begin{tabular}{lccl}
\toprule
\textbf{Quantity} & \textbf{$K_4$} & \textbf{Physics} & \textbf{Status} \\
\midrule
Spatial dimensions & 3 & 3 & Exact match \\
Time dimensions & 1 & 1 & Exact match \\
Signature & $(-,+,+,+)$ & $(-,+,+,+)$ & Exact match \\
Coupling factor & 8 & 8 & Exact match \\
$\alpha^{-1}$ & 137.036 & 137.036 & 0.00003\% \\
$m_p/m_e$ & 1836 & 1836.15 & 0.008\% \\
$m_\mu/m_e$ & 207 & 206.77 & 0.1\% \\
$m_\tau/m_e$ & 3519 & 3477 & 1.2\% \\
\bottomrule
\end{tabular}
\end{center}

\textbf{4 exact matches. 4 close matches.}

\medskip
\textit{Is this meaningful or coincidental? This document does not answer that question.}
\end{observation}

% ============================================================================
\part{Discussion}
% ============================================================================

% ============================================================================
\section{What Is Actually Proven}
\label{sec:proven-summary}
% ============================================================================

\begin{proven}
\textbf{Mathematical theorems (Agda \texttt{--safe}):}
\begin{enumerate}
  \item From ``distinction exists,'' the complete graph $K_4$ emerges uniquely
  \item $K_4$ has specific invariants: $V=4$, $E=6$, $\chi=2$, $\text{deg}=3$
  \item The Laplacian has eigenvalues $\{0,4,4,4\}$
  \item The eigenspace dimension is 3
  \item Specific combinatorial formulas produce 137, 1836, 207, 3519
  \item The entanglement identity $\chi \cdot \text{deg} = E$ is unique to $K_4$
\end{enumerate}

These are \textbf{facts about graph theory}, machine-verified.
\end{proven}

% ============================================================================
\section{What Is Not Proven}
\label{sec:not-proven}
% ============================================================================

\begin{observation}
\textbf{Not proven:}
\begin{enumerate}
  \item That $K_4$ ``is'' physical spacetime
  \item That the number 137.036 ``is'' the fine structure constant
  \item That 1836 ``is'' the proton/electron mass ratio
  \item That any of these correspondences are meaningful
  \item That the numerical agreements are not coincidental
\end{enumerate}

The claim ``mathematics determines physics'' is a \textbf{philosophical hypothesis}, not a theorem.
\end{observation}

% ============================================================================
\section{Why Not \texorpdfstring{$K_3$}{K3} or \texorpdfstring{$K_5$}{K5}?}
\label{sec:why-k4}
% ============================================================================

\begin{proven}
\textbf{Mathematical answer:} $K_4$ is the unique minimal complete graph satisfying memory saturation and closure. This is proven.
\end{proven}

\begin{observation}
\textbf{Numerical observation:} If we compute the same formulas for $K_3$ and $K_5$:

\begin{center}
\begin{tabular}{lcccc}
\toprule
\textbf{Quantity} & \textbf{$K_3$} & \textbf{$K_4$} & \textbf{$K_5$} & \textbf{Expt.} \\
\midrule
Eigenspace dim & 2 & 3 & 4 & 3 \\
$2V$ & 6 & 8 & 10 & 8 \\
``$\alpha^{-1}$'' & 31 & 137 & 266 & 137 \\
``$m_p/m_e$'' & 288 & 1836 & 8448 & 1836 \\
\bottomrule
\end{tabular}
\end{center}

Only $K_4$ values match experiment.

\medskip
\textit{This could be evidence for a deep connection, or it could be selection bias---we only notice when formulas match.}
\end{observation}

% ============================================================================
\section{Possible Interpretations}
\label{sec:interpretations}
% ============================================================================

\subsection{Interpretation 1: Coincidence}

The numbers happen to match. With enough formulas and enough constants, some will align by chance. This is the null hypothesis and cannot be ruled out.

\subsection{Interpretation 2: Selection Bias}

We found formulas that fit the data. This is a form of numerology. The fact that formulas exist does not mean they are fundamental.

\subsection{Interpretation 3: Deep Connection}

Mathematics constrains physics. The structure of distinction \textit{is} the structure of reality. This is the strongest claim and the least proven.

\begin{keyinsight}
This document presents the mathematics honestly. The interpretation is left to the reader. We do not claim Interpretation 3 is correct---only that the numerical agreements are remarkable enough to warrant attention.
\end{keyinsight}

% ============================================================================
\section{What Would Falsify This}
\label{sec:falsification}
% ============================================================================

\begin{enumerate}
  \item \textbf{Mathematical error:} Find a bug in the Agda code. This would falsify the proofs.
  \item \textbf{Alternative derivation:} Show that a different graph (not $K_4$) emerges from distinction.
  \item \textbf{Hidden parameter:} Find an adjustable parameter in the formulas.
  \item \textbf{Better numerology:} Find simpler formulas that match the constants better.
\end{enumerate}

The mathematical claims are falsifiable by finding errors in the Agda code. The physical interpretation is harder to falsify but is clearly labeled as hypothesis.

% ============================================================================
\section{Open Questions}
\label{sec:open}
% ============================================================================

\begin{enumerate}
  \item \textbf{Why Fermat primes?} The appearance of $F_2 = 17$ is unexplained.
  \item \textbf{Fractional precision:} The formula gives 137.036, experiment gives 137.035999. Where does the difference come from?
  \item \textbf{Other constants:} Can this approach predict constants we haven't yet matched?
  \item \textbf{Dynamics:} $K_4$ is static. How do equations of motion emerge?
\end{enumerate}

% ============================================================================
\section{Conclusion}
\label{sec:conclusion}
% ============================================================================

\subsection{Summary}

\begin{proven}
\textbf{What we proved:}
\begin{itemize}
  \item $K_4$ emerges uniquely from the concept of distinction
  \item $K_4$ has specific spectral and combinatorial properties
  \item These properties compute to specific numbers
\end{itemize}
All proofs are machine-verified in Agda.
\end{proven}

\begin{observation}
\textbf{What we observed:}
\begin{itemize}
  \item Those numbers match physical constants
  \item The matches are precise (0.00003\% to 1.2\%)
  \item Only $K_4$ produces matching values
\end{itemize}
No causal connection is proven.
\end{observation}

\subsection{Final Statement}

\begin{keyinsight}
The mathematics is certain. The physics is hypothesis.

We present the strongest possible mathematical evidence, but we do not claim to have derived physics from pure thought. That claim would require proof we do not have.

If you find an error, open an issue. We want to know.
\end{keyinsight}

% ============================================================================
\section{Notation Reference}
\label{sec:notation}
% ============================================================================

\begin{tabular}{ll}
\toprule
\textbf{Symbol} & \textbf{Meaning} \\
\midrule
$D_0, D_1, D_2, D_3$ & The four primordial distinctions \\
$K_4$ & Complete graph on 4 vertices \\
$V = 4$ & Vertex count \\
$E = 6$ & Edge count \\
$F = 4$ & Face count (as tetrahedron) \\
$\chi = 2$ & Euler characteristic \\
$\text{deg} = 3$ & Vertex degree \\
$\lambda = 4$ & Non-trivial Laplacian eigenvalue \\
$F_2 = 17$ & Fermat prime $2^{2^2} + 1$ \\
$L$ & Graph Laplacian matrix \\
\bottomrule
\end{tabular}

% ============================================================================
% BIBLIOGRAPHY
% ============================================================================

\begin{thebibliography}{99}

\bibitem{martinlof1972}
P. Martin-Löf, \textit{An Intuitionistic Theory of Types}, 
Twenty-Five Years of Constructive Type Theory (1972).

\bibitem{agda}
The Agda Team, \textit{Agda Documentation}, 
\url{https://agda.readthedocs.io/}

\bibitem{codata2018}
CODATA, \textit{Recommended Values of the Fundamental Physical Constants: 2018}, 
Rev. Mod. Phys. 93, 025010 (2021).

\bibitem{spencerbrown}
G. Spencer-Brown, \textit{Laws of Form}, 
Julian Press (1969).

\end{thebibliography}

\end{document}
