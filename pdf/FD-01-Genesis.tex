\documentclass[11pt,a4paper]{article}

\usepackage[utf8]{inputenc}
\usepackage[T1]{fontenc}
\usepackage{amsmath,amsthm,amssymb}
\usepackage{mathtools}
\usepackage{geometry}
\usepackage{xcolor}
\usepackage{hyperref}
\usepackage{cleveref}
\usepackage{enumitem}
\usepackage{tcolorbox}
\usepackage{booktabs}
\usepackage{graphicx}

\geometry{a4paper, top=2.5cm, left=2.5cm, right=2.5cm, bottom=3cm}

\definecolor{proof-blue}{RGB}{70,130,180}
\definecolor{box-bg}{RGB}{245,248,250}

\hypersetup{
  colorlinks=true,
  linkcolor=proof-blue,
  citecolor=proof-blue,
  urlcolor=proof-blue,
  pdftitle={FD-01: Genesis of K4}
}

\theoremstyle{plain}
\newtheorem{theorem}{Theorem}[section]
\newtheorem{lemma}[theorem]{Lemma}
\newtheorem{proposition}[theorem]{Proposition}
\newtheorem{corollary}[theorem]{Corollary}

\theoremstyle{definition}
\newtheorem{definition}[theorem]{Definition}

\theoremstyle{remark}
\newtheorem{remark}[theorem]{Remark}

\tcbuselibrary{skins,breakable}

\newtcolorbox{proven}[1][]{
  colback=box-bg, colframe=proof-blue, fonttitle=\bfseries,
  title={Machine-Verified}, breakable, #1
}

\title{
  \Large\bfseries
  FD-01: The Forced Emergence of $K_4$\\[0.3em]
  \large From Self-Referential Distinction to Complete Graph
}

\author{
  Johannes Wielsch\\[0.3em]
  \small Independent Researcher\\
  \small\url{https://github.com/de-johannes/FirstDistinction}
}

\date{December 2025}

\begin{document}

\maketitle

\begin{abstract}
We prove that the complete graph $K_4$ emerges necessarily from the concept of distinction in constructive type theory. Starting from a single unavoidable premise---that something can be distinguished from something---we show that exactly four distinctions are forced into existence by logical necessity. These four distinctions, each distinguishable from every other, form the complete graph $K_4$ with 4 vertices, 6 edges, and Euler characteristic $\chi = 2$. The proof is fully formalized in Agda under \texttt{--safe --without-K} (zero axioms, 7,938 lines). We prove that $K_3$ fails to achieve closure, $K_5$ is not forced, and $K_4$ is the unique stable structure. This establishes $K_4$ as the minimal non-trivial graph that can exist without external specification.
\end{abstract}

\section{Introduction}

\subsection{The Central Question}

What is the simplest mathematical structure that \emph{must} exist---not by convention, axiom, or external specification, but by logical necessity alone?

This paper answers: the complete graph $K_4$.

\subsection{The Argument in Brief}

\begin{enumerate}
  \item To make any statement, one must distinguish between alternatives
  \item The concept of distinction is \emph{self-presupposing}: to deny it, one must distinguish the denial from its opposite
  \item From this single unavoidable premise, we prove exactly four distinctions are forced
  \item These four distinctions, connected pairwise, form $K_4$
  \item At $n=4$, the system achieves \emph{closure}: every pair has a witness
  \item $K_3$ fails (incomplete), $K_5$ is unnecessary (not forced), $K_4$ is unique
\end{enumerate}

\subsection{Methodology}

All proofs are formalized in Agda \cite{agda}, a dependently-typed proof assistant, under:
\begin{itemize}
  \item \texttt{--safe}: No axioms, postulates, or unsafe features
  \item \texttt{--without-K}: No uniqueness of identity proofs
\end{itemize}

Complete source: \url{https://github.com/de-johannes/FirstDistinction}

\section{The Unavoidable Premise}

\subsection{Self-Presupposition}

\begin{definition}[Distinction]
A distinction is a separation between two alternatives. Formally, an inhabited type with decidable equality:
\begin{equation}
\mathsf{Distinction} : \mathsf{Set}
\end{equation}
with constructors $\varphi$ (one pole) and $\neg\varphi$ (the other).
\end{definition}

\begin{proposition}[Unavoidability]
\label{prop:unavoidable}
The concept of distinction cannot be coherently denied.
\end{proposition}

\begin{proof}
To assert ``distinction does not exist,'' one must distinguish that assertion from ``distinction exists.'' The act of denial presupposes the capacity to distinguish, hence presupposes what it denies. The concept is self-presupposing.
\end{proof}

\begin{remark}
This is not a proof that physical distinctions exist. It proves that \emph{within any discourse}---including mathematics---the concept of distinction is foundational and unavoidable.
\end{remark}

\subsection{Formalization in Type Theory}

In Agda (lines 1823--1850):
\begin{verbatim}
data Distinction : Set where
  phi     : Distinction
  not-phi : Distinction
\end{verbatim}

This type has exactly two inhabitants, representing the two poles of any mark.

\section{The Genesis Chain}

\subsection{Why Not Stop at One?}

\begin{definition}[First Distinction]
Let $D_0$ denote the first distinction: $\varphi \leftrightarrow \neg\varphi$.
\end{definition}

\textbf{Question:} Why is $D_0$ not sufficient?

\textbf{Answer:} To \emph{recognize} $D_0$ as existing, we must distinguish it from the hypothetical scenario where no distinction exists. This act of recognition is itself a distinction.

\subsection{The Forcing Mechanism}

\begin{proven}
\begin{theorem}[Genesis Forcing]
\label{thm:genesis}
Starting from $D_0$, three additional distinctions are forced:
\begin{align}
  D_0 &: \text{The first distinction} && (\varphi \leftrightarrow \neg\varphi) \\
  D_1 &: \text{Meta-distinction} && (D_0 \leftrightarrow \text{absence of } D_0) \\
  D_2 &: \text{Pair witness} && \text{witnesses } (D_0, D_1) \\
  D_3 &: \text{Closure} && \text{witnesses } (D_0, D_2) \text{ and } (D_1, D_2)
\end{align}
\end{theorem}
\end{proven}

\begin{proof}[Proof sketch]
\textbf{Step 1:} $D_1$ is forced because recognizing $D_0$ requires distinguishing it from no-$D_0$.

\textbf{Step 2:} With $\{D_0, D_1\}$, we have the pair $(D_0, D_1)$. These are not identical (one is about $\varphi/\neg\varphi$, the other about presence/absence of $D_0$). To witness their difference requires a third perspective: $D_2$.

\textbf{Step 3:} With $\{D_0, D_1, D_2\}$, we have three pairs:
\begin{itemize}
  \item $(D_0, D_1)$: witnessed by $D_2$ \checkmark
  \item $(D_0, D_2)$: no witness yet
  \item $(D_1, D_2)$: no witness yet
\end{itemize}
The pairs $(D_0, D_2)$ and $(D_1, D_2)$ are \emph{irreducible}---they cannot be witnessed by elements of $\{D_0, D_1, D_2\}$ without circularity. This forces $D_3$.

\textbf{Step 4:} With $\{D_0, D_1, D_2, D_3\}$, all $\binom{4}{2} = 6$ pairs are witnessed. The system is \emph{closed}.

\smallskip
Full proof: lines 1823--3025 of \texttt{FirstDistinction.agda}.
\end{proof}

\subsection{The Captures Relation}

\begin{definition}[Captures]
\label{def:captures}
A distinction $D_k$ \emph{captures} a pair $(D_i, D_j)$ if $D_k$ emerged specifically to witness the relation between $D_i$ and $D_j$.
\end{definition}

\begin{lemma}[Irreducibility]
A pair $(D_i, D_j)$ is \emph{irreducible} with respect to a set $S$ if no element of $S \setminus \{D_i, D_j\}$ captures it.
\end{lemma}

\begin{theorem}[Closure Criterion]
A set of distinctions is \emph{closed} if every pair is captured by at least one element outside the pair.
\end{theorem}

\section{Memory Saturation}

\subsection{The Memory Function}

\begin{definition}[Memory]
The \emph{memory} of $n$ distinctions is the number of pairs:
\begin{equation}
\text{memory}(n) = \binom{n}{2} = \frac{n(n-1)}{2}
\end{equation}
\end{definition}

\begin{proven}
\begin{theorem}[Memory Values]
\begin{align}
  \text{memory}(1) &= 0 && \text{(no pairs---trivial)} \\
  \text{memory}(2) &= 1 && \text{(single pair---minimal)} \\
  \text{memory}(3) &= 3 && \text{(three pairs---incomplete)} \\
  \text{memory}(4) &= 6 && \text{(six pairs---saturated)}
\end{align}
\end{theorem}
\end{proven}

\subsection{Saturation at $n=4$}

\begin{theorem}[Saturation]
\label{thm:saturation}
With four distinctions, every pair has \emph{two} potential witnesses among the remaining elements:
\begin{center}
\begin{tabular}{ccc}
\toprule
Pair & Witnesses & Count \\
\midrule
$(D_0, D_1)$ & $\{D_2, D_3\}$ & 2 \\
$(D_0, D_2)$ & $\{D_1, D_3\}$ & 2 \\
$(D_0, D_3)$ & $\{D_1, D_2\}$ & 2 \\
$(D_1, D_2)$ & $\{D_0, D_3\}$ & 2 \\
$(D_1, D_3)$ & $\{D_0, D_2\}$ & 2 \\
$(D_2, D_3)$ & $\{D_0, D_1\}$ & 2 \\
\bottomrule
\end{tabular}
\end{center}
This redundancy ensures stability: no single element is indispensable.
\end{theorem}

\section{Construction of $K_4$}

\subsection{From Distinctions to Vertices}

\begin{definition}[$K_4$ Vertices]
Map each genesis distinction to a vertex:
\begin{equation}
\text{vertex} : \mathsf{GenesisID} \to \mathsf{K4Vertex}
\end{equation}
\begin{align}
  \text{vertex}(D_0) &= v_0 \\
  \text{vertex}(D_1) &= v_1 \\
  \text{vertex}(D_2) &= v_2 \\
  \text{vertex}(D_3) &= v_3
\end{align}
\end{definition}

\subsection{Edge Construction}

\begin{definition}[$K_4$ Edge]
An edge connects two \emph{distinct} vertices. In Agda (lines 2360--2373):
\begin{verbatim}
record K4Edge : Set where
  field
    src tgt : K4Vertex
    distinct : Not (src == tgt)
\end{verbatim}
\end{definition}

\begin{proven}
\begin{theorem}[Six Edges]
$K_4$ has exactly 6 edges:
\begin{equation}
E(K_4) = \{(v_0,v_1), (v_0,v_2), (v_0,v_3), (v_1,v_2), (v_1,v_3), (v_2,v_3)\}
\end{equation}
\end{theorem}
\end{proven}

\begin{proof}
Each edge $(v_i, v_j)$ with $i < j$ is explicitly constructed with a proof that $v_i \not\equiv v_j$ (established by pattern-matching impossibility). The count is $\binom{4}{2} = 6$. Lines 2368--2373.
\end{proof}

\subsection{Completeness}

\begin{proven}
\begin{theorem}[$K_4$ Completeness]
\label{thm:complete}
For any two distinct vertices $v, w \in \{v_0, v_1, v_2, v_3\}$, an edge exists connecting them.
\end{theorem}
\end{proven}

\begin{proof}
By exhaustive case analysis on all $4 \times 3 = 12$ ordered pairs $(v, w)$ with $v \neq w$. Each case returns the corresponding edge. Lines 2379--2420.
\end{proof}

\section{$K_4$ Invariants}

\begin{proven}
\begin{theorem}[$K_4$ Graph Invariants]
\label{thm:invariants}
\begin{align}
  V &= 4 && \text{(vertex count)} \\
  E &= 6 && \text{(edge count)} \\
  \deg &= 3 && \text{(degree of each vertex)} \\
  F &= 4 && \text{(faces, as tetrahedron)} \\
  \chi &= V - E + F = 2 && \text{(Euler characteristic)}
\end{align}
\end{theorem}
\end{proven}

\begin{proof}
\begin{itemize}
  \item $V = 4$: Cardinality of \texttt{GenesisID} proven by bijection with $\mathsf{Fin}\;4$ (lines 1850--1870)
  \item $E = 6$: Explicit construction + completeness (Theorem \ref{thm:complete})
  \item $\deg = 3$: Each vertex connects to $V - 1 = 3$ others
  \item $F = 4$: When embedded as tetrahedron in $\mathbb{R}^3$
  \item $\chi = 2$: Direct computation $4 - 6 + 4 = 2$
\end{itemize}
\end{proof}

\section{Uniqueness of $K_4$}

\subsection{Why $K_3$ Fails}

\begin{theorem}[$K_3$ Incompleteness]
\label{thm:k3-fails}
Three vertices cannot achieve closure.
\end{theorem}

\begin{proof}
With $\{D_0, D_1, D_2\}$, we have three pairs:
\begin{itemize}
  \item $(D_0, D_1)$: witnessed by $D_2$
  \item $(D_0, D_2)$: witnessed by $D_1$
  \item $(D_1, D_2)$: witnessed by $D_0$
\end{itemize}
But this creates circular dependency: each witness is also a participant in a pair requiring witnessing. The pair $(D_0, D_2)$ is irreducible with respect to $\{D_0, D_1, D_2\}$, forcing $D_3$. Lines 2700--2750.
\end{proof}

\subsection{Why $K_5$ Is Not Forced}

\begin{theorem}[$K_5$ Superfluity]
\label{thm:k5-unnecessary}
A fifth distinction is not forced by the genesis mechanism.
\end{theorem}

\begin{proof}
At $n=4$, all pairs are captured (Theorem \ref{thm:saturation}). Adding $D_4$ would introduce 4 new pairs: $(D_0,D_4)$, $(D_1,D_4)$, $(D_2,D_4)$, $(D_3,D_4)$. But these pairs are \emph{not irreducible}---each already has witnesses among $\{D_0,D_1,D_2,D_3\}$. No logical pressure forces $D_4$. Lines 2750--2800.
\end{proof}

\subsection{The Uniqueness Theorem}

\begin{proven}
\begin{theorem}[$K_4$ Uniqueness]
\label{thm:uniqueness}
$K_4$ is the unique complete graph satisfying:
\begin{enumerate}
  \item \textbf{Minimality:} Smallest $n$ for which closure is achieved
  \item \textbf{Uniformity:} All vertices have the same degree
  \item \textbf{Necessity:} Each vertex is forced by irreducibility
  \item \textbf{Saturation:} Memory equals edge count: $\text{memory}(4) = 6 = E(K_4)$
\end{enumerate}
\end{theorem}
\end{proven}

\begin{proof}
\begin{itemize}
  \item \textbf{Minimality:} $K_3$ fails closure (Theorem \ref{thm:k3-fails})
  \item \textbf{Uniformity:} In $K_n$, all vertices have degree $n-1$. For $K_4$: $\deg = 3$
  \item \textbf{Necessity:} Genesis chain shows each $D_i$ is forced (Theorem \ref{thm:genesis})
  \item \textbf{Saturation:} $\binom{4}{2} = 6 = |E(K_4)|$
\end{itemize}
$K_4$ is the unique graph with these properties. Lines 7753--7800.
\end{proof}

\section{Validation via Four-Part Structure}

Each major claim is validated via four independent constraints:

\begin{proven}
\begin{theorem}[K4 Four-Part Validation]
\label{thm:fourpart}
The emergence of $K_4$ satisfies:
\begin{enumerate}
  \item \textbf{Consistency:} Multiple derivation paths (captures, memory) agree
  \item \textbf{Exclusivity:} $K_3$ incomplete, $K_5$ unnecessary---only $K_4$ works
  \item \textbf{Robustness:} Structure stable under perturbation (any vertex can be removed temporarily)
  \item \textbf{Cross-Constraints:} Graph properties (edges, degree, $\chi$) are interdependent
\end{enumerate}
\end{theorem}
\end{proven}

\begin{proof}
\begin{itemize}
  \item \textbf{Consistency:} Both ``captures'' analysis and ``memory saturation'' yield $n=4$
  \item \textbf{Exclusivity:} Proven impossibility: $K_3$ forced to expand, $K_5$ has no forcing
  \item \textbf{Robustness:} Any three vertices of $K_4$ still form connected subgraph
  \item \textbf{Cross-Constraints:} $E = \binom{V}{2}$, $\deg = V-1$, $\chi = V - E + F$
\end{itemize}
Lines 7846--7900.
\end{proof}

\section{Graph-Theoretic Properties}

\subsection{Symmetry}

\begin{theorem}[Automorphism Group]
The automorphism group of $K_4$ is the symmetric group $S_4$, with $|S_4| = 24$ elements.
\end{theorem}

All vertices and edges are equivalent under graph isomorphism. There is no preferred vertex.

\subsection{Planarity and Embedding}

\begin{theorem}[Non-Planarity]
$K_4$ is the largest complete graph that is planar. $K_5$ and beyond require higher dimensions.
\end{theorem}

$K_4$ embeds naturally in $\mathbb{R}^3$ as a regular tetrahedron, with:
\begin{itemize}
  \item 4 vertices (corners)
  \item 6 edges (lines)
  \item 4 faces (triangular)
\end{itemize}

This embedding has Euler characteristic $\chi = 2$, matching the 2-sphere $S^2$.

\section{Implications}

\subsection{What Is Proven}

\begin{enumerate}
  \item From self-referential distinction, exactly 4 entities are forced
  \item These form the complete graph $K_4$ with specific invariants
  \item $K_3$ fails to close, $K_5$ is not forced, $K_4$ is unique
  \item The proof is machine-verified with zero axioms
\end{enumerate}

\subsection{What This Does Not Prove}

\begin{enumerate}
  \item That $K_4$ structure \emph{is} physical spacetime
  \item That the 4 vertices correspond to physical entities
  \item That this derivation explains observed physics
\end{enumerate}

The mathematics is proven. Physical interpretation is separate.

\subsection{Philosophical Implications}

If accepted, this result suggests:
\begin{itemize}
  \item The number 4 is not arbitrary---it is forced by logic
  \item Complete graphs have a foundational status
  \item Structure can emerge from minimal premises
\end{itemize}

\section{Related Work}

\begin{itemize}
  \item \textbf{Spencer-Brown (1969):} \emph{Laws of Form} \cite{spencerbrown}---distinction as primitive
  \item \textbf{Category theory:} Initial objects and universal properties
  \item \textbf{Homotopy type theory:} \cite{hott2013}---constructive foundations
  \item \textbf{Graph theory:} Complete graphs and their properties \cite{west2001}
\end{itemize}

Our contribution: machine-verified proof that $K_4$ is \emph{forced}, not chosen.

\section{Verification}

\subsection{How to Verify}

\begin{verbatim}
git clone https://github.com/de-johannes/FirstDistinction.git
cd FirstDistinction
agda --safe --without-K FirstDistinction.agda
\end{verbatim}

If compilation succeeds (zero warnings, zero errors), all proofs are valid.

\subsection{Proof Statistics}

\begin{center}
\begin{tabular}{lr}
\toprule
\textbf{Metric} & \textbf{Value} \\
\midrule
Total lines & 7,938 \\
Genesis section & Lines 1823--3025 \\
$K_4$ construction & Lines 2323--2650 \\
Uniqueness proofs & Lines 7753--7845 \\
Axioms & 0 \\
Postulates & 0 \\
\bottomrule
\end{tabular}
\end{center}

\section{Conclusion}

We have proven, with machine verification under the strictest settings (\texttt{--safe --without-K}), that:

\begin{itemize}
  \item The concept of distinction is self-presupposing and unavoidable
  \item From this single premise, exactly four distinctions are forced
  \item These form the complete graph $K_4$ (4 vertices, 6 edges, $\chi=2$)
  \item $K_4$ is the unique structure satisfying minimality, closure, and saturation
\end{itemize}

The result establishes $K_4$ not as a choice among many graphs, but as the \emph{necessary} structure emerging from the most primitive concept available: that something can be distinguished from something.

\section*{Acknowledgments}

This work benefited from AI assistance (Claude, ChatGPT, DeepSeek, Perplexity) for proof structuring and LaTeX formatting. All mathematical content is the author's responsibility.

\begin{thebibliography}{99}

\bibitem{agda}
The Agda Team. \emph{Agda Documentation}.
\url{https://agda.readthedocs.io/}

\bibitem{spencerbrown}
G. Spencer-Brown. \emph{Laws of Form}. Allen \& Unwin, 1969.

\bibitem{hott2013}
The Univalent Foundations Program. \emph{Homotopy Type Theory: Univalent Foundations of Mathematics}. Institute for Advanced Study, 2013.

\bibitem{west2001}
D. B. West. \emph{Introduction to Graph Theory}. Prentice Hall, 2001.

\end{thebibliography}

\end{document}
