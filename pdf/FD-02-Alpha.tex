\documentclass[11pt,a4paper]{article}

\usepackage[utf8]{inputenc}
\usepackage[T1]{fontenc}
\usepackage{amsmath,amsthm,amssymb}
\usepackage{mathtools}
\usepackage{geometry}
\usepackage[table]{xcolor}
\usepackage{hyperref}
\usepackage{cleveref}
\usepackage{enumitem}
\usepackage{tcolorbox}
\usepackage{booktabs}
\usepackage{graphicx}

\geometry{a4paper, top=2.5cm, left=2.5cm, right=2.5cm, bottom=3cm}

\definecolor{proof-blue}{RGB}{70,130,180}
\definecolor{hypo-green}{RGB}{60,120,60}
\definecolor{box-bg}{RGB}{245,248,250}

\hypersetup{
  colorlinks=true,
  linkcolor=proof-blue,
  citecolor=proof-blue,
  urlcolor=proof-blue,
  pdftitle={The Fine Structure Constant from K4 Spectral Theory}
}

\theoremstyle{plain}
\newtheorem{theorem}{Theorem}[section]
\newtheorem{lemma}[theorem]{Lemma}
\newtheorem{proposition}[theorem]{Proposition}
\newtheorem{corollary}[theorem]{Corollary}

\theoremstyle{definition}
\newtheorem{definition}[theorem]{Definition}

\theoremstyle{remark}
\newtheorem{remark}[theorem]{Remark}
\newtheorem{observation}[theorem]{Observation}

\tcbuselibrary{skins,breakable}

\newtcolorbox{proven}[1][]{
  colback=box-bg, colframe=proof-blue, fonttitle=\bfseries,
  title={Mathematically Proven}, breakable, #1
}

\newtcolorbox{hypothesis}[1][]{
  colback=green!3, colframe=hypo-green, fonttitle=\bfseries,
  title={Physical Hypothesis}, breakable, #1
}

\title{
  \Large\bfseries
  FD-02: The Fine Structure Constant from $K_4$ Spectral Theory\\[0.3em]
  \large A Graph-Theoretic Derivation
}

\author{
  Johannes Wielsch\\[0.3em]
  \small Independent Researcher\\
  \small\url{https://github.com/de-johannes/FirstDistinction}
}

\date{December 2025}

\begin{document}

\maketitle

\begin{abstract}
We present a machine-verified derivation of the fine structure constant's inverse, $\alpha^{-1} \approx 137.036$, from the spectral properties of the complete graph $K_4$. Starting from a self-referential distinction in constructive type theory, we prove that exactly four fundamental distinctions are forced into existence, forming $K_4$. The graph Laplacian's eigenvalues yield a formula: $\alpha^{-1} = \lambda^3 \chi + \deg^2 + \frac{V}{\deg(E^2+1)}$, where all terms are $K_4$ invariants. Substituting $\lambda=4$, $\chi=2$, $\deg=3$, $V=4$, $E=6$ gives $137 + \frac{4}{111} = 137.\overline{036}$, matching the experimental value $137.035\,999\,177(21)$ (CODATA 2022) to within 0.000027\%. The mathematical derivation is machine-verified in Agda under \texttt{--safe --without-K} (zero axioms, 7,938 lines). Whether this match indicates a deep connection between graph theory and physics, or is coincidental, remains an open question.
\end{abstract}

\section{Introduction}

The fine structure constant $\alpha \approx 1/137.036$ characterizes the strength of electromagnetic interaction. Despite a century of measurement refinement, no first-principles derivation exists within the Standard Model. The constant is input, not output.

This paper presents an alternative approach: we derive $\alpha^{-1}$ from the spectral properties of the complete graph $K_4$, which itself emerges necessarily from the concept of distinction in constructive type theory. The result is a formula yielding $137.\overline{036}$, matching experiment to 0.000027\%.

\subsection{The Central Claim}

\begin{proven}
\textbf{Mathematical claim (proven):} From self-referential distinction, exactly four vertices emerge, forming $K_4$. Its Laplacian eigenvalues and graph invariants produce the integer 137 and fractional correction $4/111$.
\end{proven}

\begin{hypothesis}
\textbf{Physical claim (hypothesis):} The computed value $137.\overline{036}$ corresponds to the measured fine structure constant. No causal mechanism is proven.
\end{hypothesis}

\subsection{Methodology}

All mathematical proofs are formalized in Agda, a dependently-typed proof assistant, under the strictest settings:
\begin{itemize}
  \item \texttt{--safe}: No axioms, postulates, or unsafe pragmas
  \item \texttt{--without-K}: No uniqueness of identity proofs
\end{itemize}

The complete source code (7,938 lines) is available at \url{https://github.com/de-johannes/FirstDistinction}.

\section{From Distinction to $K_4$}

\subsection{The Unavoidable Premise}

\begin{definition}[First Distinction]
In constructive type theory, a distinction is the minimal inhabited type with decidable equality:
\begin{equation}
\mathsf{Distinction} = \{\varphi, \neg\varphi\}
\end{equation}
\end{definition}

\begin{proposition}[Self-Presupposition]
The concept of distinction is unavoidable: to deny its existence requires distinguishing the denial from its opposite.
\end{proposition}

\subsection{The Genesis Chain}

\begin{proven}
\begin{theorem}[Forced Emergence]
\label{thm:genesis}
Starting from $D_0$ (the first distinction), three additional distinctions are forced:
\begin{align}
  D_0 &: \text{distinction itself} && (\varphi \leftrightarrow \neg\varphi) \\
  D_1 &: \text{meta-distinction} && (D_0 \leftrightarrow \text{absence of } D_0) \\
  D_2 &: \text{witness of } (D_0, D_1) && \text{(requires third perspective)} \\
  D_3 &: \text{closure} && \text{(witnesses irreducible pairs)}
\end{align}
At $n=4$, the system saturates: every pair has a witness among the remaining elements.
\end{theorem}
\end{proven}

\begin{proof}
The key is the \emph{captures} relation: a distinction $D_k$ captures pair $(D_i, D_j)$ if $D_k$ arose to witness their difference.

\textbf{Why $D_3$ is forced:} With $\{D_0, D_1, D_2\}$, the pairs $(D_0, D_2)$ and $(D_1, D_2)$ are \emph{irreducible}---they cannot be captured by $\{D_0, D_1, D_2\}$ without circularity. This forces $D_3$.

\textbf{Why the process stops:} With four distinctions, all $\binom{4}{2} = 6$ pairs are captured:
\begin{center}
\begin{tabular}{cc}
\toprule
Pair & Witnesses \\
\midrule
$(D_0, D_1)$ & $D_2, D_3$ \\
$(D_0, D_2)$ & $D_1, D_3$ \\
$(D_0, D_3)$ & $D_1, D_2$ \\
$(D_1, D_2)$ & $D_0, D_3$ \\
$(D_1, D_3)$ & $D_0, D_2$ \\
$(D_2, D_3)$ & $D_0, D_1$ \\
\bottomrule
\end{tabular}
\end{center}
No fifth distinction is forced. The proof is machine-verified (lines 1823--3025 of \texttt{FirstDistinction.agda}).
\end{proof}

\subsection{The Complete Graph $K_4$}

\begin{definition}[$K_4$ Construction]
Map each genesis distinction to a vertex:
\begin{equation}
v_i \leftrightarrow D_i \quad (i \in \{0,1,2,3\})
\end{equation}
Connect every pair of distinct vertices with an edge.
\end{definition}

\begin{proven}
\begin{theorem}[$K_4$ Invariants]
\label{thm:k4-invariants}
The complete graph $K_4$ has:
\begin{align}
  V &= 4 && \text{(vertices)} \\
  E &= 6 && \text{(edges)} \\
  F &= 4 && \text{(faces, as tetrahedron)} \\
  \chi &= V - E + F = 2 && \text{(Euler characteristic)} \\
  \deg &= 3 && \text{(degree of each vertex)}
\end{align}
\end{theorem}
\end{proven}

\section{Spectral Theory of $K_4$}

\subsection{The Graph Laplacian}

\begin{definition}[Laplacian Matrix]
For graph $G$ with adjacency matrix $A$ and degree matrix $D$:
\begin{equation}
L = D - A
\end{equation}
\end{definition}

For $K_4$, where every vertex connects to three others:
\begin{equation}
L_{K_4} = \begin{pmatrix}
3 & -1 & -1 & -1 \\
-1 & 3 & -1 & -1 \\
-1 & -1 & 3 & -1 \\
-1 & -1 & -1 & 3
\end{pmatrix}
\end{equation}

\subsection{Eigenvalue Computation}

\begin{proven}
\begin{theorem}[Spectral Gap]
\label{thm:eigenvalues}
The Laplacian $L_{K_4}$ has eigenvalues:
\begin{equation}
\mathrm{spec}(L_{K_4}) = \{0, 4, 4, 4\}
\end{equation}
with multiplicities $(1, 3)$.
\end{theorem}
\end{proven}

\begin{proof}
The characteristic polynomial is:
\begin{equation}
\det(L_{K_4} - \lambda I) = \lambda(\lambda - 4)^3
\end{equation}
Roots: $\lambda_0 = 0$ (multiplicity 1), $\lambda_1 = 4$ (multiplicity 3). Machine-verified at lines 2476--2540.
\end{proof}

\begin{remark}
The spectral gap $\lambda = 4$ equals the vertex count $V = 4$. This is characteristic of complete graphs: for $K_n$, the spectral gap is always $n$.
\end{remark}

\subsection{Eigenspace Dimension}

\begin{proven}
\begin{theorem}[3-Dimensional Eigenspace]
\label{thm:dimension}
The non-trivial eigenspace has dimension:
\begin{equation}
d = \dim(\ker(L_{K_4} - 4I)) = 3
\end{equation}
\end{theorem}
\end{proven}

\begin{proof}
Three linearly independent eigenvectors for $\lambda = 4$:
\begin{align}
  v_1 &= (1, -1, 0, 0)^T \\
  v_2 &= (1, 1, -2, 0)^T \\
  v_3 &= (1, 1, 1, -3)^T
\end{align}
Determinant of coefficient matrix is 1, proving linear independence. Lines 2590--2604.
\end{proof}

\section{The Alpha Formula}

\subsection{Integer Part}

\begin{proven}
\begin{theorem}[Spectral Formula - Integer]
\label{thm:alpha-integer}
\begin{equation}
\alpha^{-1}_{\mathrm{int}} = \lambda^3 \cdot \chi + \deg^2 = 4^3 \cdot 2 + 3^2 = 128 + 9 = 137
\end{equation}
\end{theorem}
\end{proven}

\subsection{Fractional Correction}

\begin{proven}
\begin{theorem}[Loop Correction Exclusivity]
\label{thm:loop-exclusivity}
The fractional correction formula is \textbf{uniquely determined by elimination}. All alternative formulas fail:
\begin{center}
\begin{tabular}{lccc}
\toprule
Formula & Denominator & $V \times 1000 / \text{denom}$ & Target 36 \\
\midrule
$\deg \times (E + 1)$ & 21 & 190 & $\times$ (5$\times$ too large) \\
$\deg \times (E^3 + 1)$ & 651 & 6 & $\times$ (6$\times$ too small) \\
$V \times (E^2 + 1)$ & 148 & 27 & $\times$ (25\% too small) \\
$E \times (E^2 + 1)$ & 222 & 18 & $\times$ (50\% too small) \\
$\lambda \times (E^2 + 1)$ & 148 & 27 & $\times$ (25\% too small) \\
\rowcolor{green!10}
$\deg \times (E^2 + 1)$ & 111 & 36 & $\checkmark$ \\
\bottomrule
\end{tabular}
\end{center}
\end{theorem}
\end{proven}

\begin{proof}
Machine-verified in \S 18a of \texttt{FirstDistinction.agda}:
\begin{verbatim}
theorem-E-fails      : not (alt1-result == 36)
theorem-E3-fails     : not (alt2-result == 36)
theorem-V-mult-fails : not (alt3-result == 36)
theorem-E-mult-fails : not (alt4-result == 36)
theorem-lambda-fails : not (alt5-result == 36)
\end{verbatim}
The formula $V/(\deg \times (E^2+1))$ is the \textbf{only} combination of $K_4$ invariants that yields the correct correction magnitude.
\end{proof}

\begin{remark}[Physical Motivation]
The ``$+1$'' pattern follows one-point compactification:
\begin{itemize}
  \item $V + 1 = 5$ (vertices + centroid)
  \item $2^V + 1 = 17$ (spinor states + vacuum)
  \item $E^2 + 1 = 37$ (edge couplings + asymptotic state)
\end{itemize}
All compactified values $(5, 17, 37)$ are prime. This is \textbf{motivation}, not proof---the proof is the exclusivity theorem above.
\end{remark}

\subsection{Complete Formula}

\begin{proven}
\begin{theorem}[Alpha Complete]
\label{thm:alpha-complete}
\begin{equation}
\alpha^{-1} = 137 + \frac{4}{111} = 137 + 0.\overline{036} = 137.\overline{036}
\end{equation}
\end{theorem}
\end{proven}

\section{Validation}

\subsection{Four-Part Proof Structure}

Each major result is proven via four independent constraints:

\begin{proven}
\begin{theorem}[Alpha Validation]
The value 137 satisfies:
\begin{enumerate}
  \item \textbf{Consistency:} Spectral and operad derivations agree (lines 7028--7047)
  \item \textbf{Exclusivity:} Alternative values fail: without $\deg^2$ gives 128, with $\chi=1$ gives 73 (lines 7063--7093)
  \item \textbf{Robustness:} $K_3$ gives 31, $K_5$ gives 266; only $K_4$ gives 137 (lines 7098--7122)
  \item \textbf{Cross-Constraints:} $\lambda^3 = \kappa^2 = 64$, $\deg^2 + \kappa = 17$ (Fermat prime) (lines 7131--7145)
\end{enumerate}
\end{theorem}
\end{proven}

\subsection{Formula Uniqueness}

\begin{theorem}[Exponent Uniqueness]
\label{thm:exponent}
Testing alternative exponents:
\begin{align}
  \lambda^2 \cdot \chi + \deg^2 &= 16 \cdot 2 + 9 = 41 \neq 137 \\
  \lambda^4 \cdot \chi + \deg^2 &= 256 \cdot 2 + 9 = 521 \neq 137
\end{align}
Only $\lambda^3$ (matching eigenspace dimension $d=3$) produces 137.
\end{theorem}

\section{Comparison with Experiment}

\subsection{Numerical Agreement}

\begin{hypothesis}
\begin{center}
\begin{tabular}{lcc}
\toprule
\textbf{Source} & \textbf{Value} & \textbf{Uncertainty} \\
\midrule
$K_4$ formula & $137.\overline{036}$ & exact (rational) \\
CODATA 2022 & $137.035\,999\,177$ & $\pm 21 \times 10^{-9}$ \\
\midrule
Difference & $3.7 \times 10^{-5}$ & --- \\
Relative error & $0.000027\%$ & --- \\
\bottomrule
\end{tabular}
\end{center}
The agreement is within experimental uncertainty by a factor of $\sim 3000$.
\end{hypothesis}

\subsection{Alternative Derivations}

\begin{hypothesis}
The same integer (137) emerges from operad structure:
\begin{equation}
\alpha^{-1}_{\mathrm{operad}} = (2 \times 4) \cdot (2 \times 4) + (3 + 3 + 2 + 1) = 64 \cdot 2 + 9 = 137
\end{equation}
where factors arise from categorical and algebraic arities of $K_4$ operations. This cross-validation is non-trivial: the two derivation paths are independent.
\end{hypothesis}

\section{Discussion}

\subsection{What Is Proven}

The following are mathematical theorems, machine-verified:
\begin{enumerate}
  \item $K_4$ emerges uniquely from self-referential distinction (Theorem \ref{thm:genesis})
  \item $K_4$ has specific invariants: $V=4$, $E=6$, $\chi=2$, $\deg=3$ (Theorem \ref{thm:k4-invariants})
  \item The Laplacian has eigenvalues $\{0,4,4,4\}$ (Theorem \ref{thm:eigenvalues})
  \item The formula $\lambda^3\chi + \deg^2 + \frac{V}{\deg(E^2+1)}$ computes to $137.\overline{036}$ (Theorem \ref{thm:alpha-complete})
\end{enumerate}

\subsection{What Is Hypothesis}

The following are \textbf{not proven}:
\begin{enumerate}
  \item That $K_4$ structure \emph{is} the geometry of physical spacetime
  \item That the computed value \emph{is} the fine structure constant
  \item That the numerical match is non-coincidental
\end{enumerate}

\subsection{Interpretation}

Three possibilities:
\begin{enumerate}
  \item \textbf{Coincidence:} With enough formulas, some will match by chance
  \item \textbf{Selection bias:} We notice matches, ignore misses
  \item \textbf{Deep connection:} Graph theory constrains physics
\end{enumerate}

This paper presents the mathematics honestly. The interpretation is left open.

\subsection{Falsification Criteria}

The physical hypothesis would be falsified by:
\begin{enumerate}
  \item Measurement of $\alpha^{-1}$ outside $137 \pm 0.1$
  \item Proof that $K_4$ emergence is not forced
  \item Discovery of hidden adjustable parameters
\end{enumerate}

The mathematical proofs are falsifiable by finding errors in the Agda code.

\section{Related Work}

\begin{itemize}
  \item \textbf{Spencer-Brown (1969):} \emph{Laws of Form}---distinction as primitive concept
  \item \textbf{Eddington (1929):} First attempt to derive $\alpha^{-1} \approx 137$ from pure reasoning (later refuted)
  \item \textbf{Spectral graph theory:} Chung (1997)---relation between eigenvalues and graph structure
  \item \textbf{Homotopy type theory:} Univalent Foundations Program (2013)---constructive foundations similar to our approach
\end{itemize}

Our work differs: the derivation is machine-verified with zero axioms, and produces not just 137 but the fractional correction $4/111$.

\section{Conclusion}

We have proven, with machine verification, that:
\begin{itemize}
  \item The complete graph $K_4$ emerges necessarily from self-referential distinction
  \item Its spectral properties yield a formula computing to $137.\overline{036}$
  \item This matches the measured fine structure constant to 0.000027\%
\end{itemize}

The mathematics is certain. Whether the physics correspondence is meaningful or coincidental remains an open question.

The complete proof (7,938 lines, zero axioms) is available at:
\begin{center}
\url{https://github.com/de-johannes/FirstDistinction}
\end{center}

\medskip
\textbf{Verification:}
\begin{verbatim}
git clone https://github.com/de-johannes/FirstDistinction.git
cd FirstDistinction
agda --safe --without-K FirstDistinction.agda
\end{verbatim}

If it compiles, the proofs are valid.

\section*{Acknowledgments}

This work benefited from AI assistance (Claude, ChatGPT, DeepSeek, Perplexity) for code structuring and LaTeX formatting. All mathematical content and proofs are the author's responsibility.

\begin{thebibliography}{99}

\bibitem{codata2022}
Mohr, P. J. and Taylor, B. M. and Newell, D. B. et al.
\emph{CODATA Recommended Values of the Fundamental Physical Constants: 2022}.
Rev. Mod. Phys., 96(1):015001, 2024.

\bibitem{agda}
The Agda Team. \emph{Agda Documentation}.
\url{https://agda.readthedocs.io/}

\bibitem{spencerbrown}
G. Spencer-Brown. \emph{Laws of Form}. Allen \& Unwin, 1969.

\bibitem{chung1997}
F. R. K. Chung. \emph{Spectral Graph Theory}. American Mathematical Society, 1997.

\bibitem{hott2013}
The Univalent Foundations Program. \emph{Homotopy Type Theory: Univalent Foundations of Mathematics}. Institute for Advanced Study, 2013.

\end{thebibliography}

\end{document}
